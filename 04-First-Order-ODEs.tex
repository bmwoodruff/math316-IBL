
\noindent This chapter covers the following ideas.

\begin{enumerate}
\item Be able to interpret the basic vocabulary of differential equations. In particular, interpret the terms ordinary differential equation (ODE), initial value, initial value problem (IVP), general solution, and particular solution.
\item Identify and solve separable and exact ODEs. 
\item Use integrating factors and substitution to solve additional ODEs.
\item Use the three step modeling process (express, solve, and interpret) to analyze exponential growth and decay, Newton's law of cooling, mixing, and the logistics equation. 
\item Use Laplace transforms to solve first order ODEs.
\end{enumerate}

The problems below come from Schaum's Outlines \textit{Differential Equations} by Richard Bronson. If you are struggling with a topic from the problem set, please use this list as a guideline to find related problems.

\begin{center}

\begin{tabular}{|l|c|l|l|l|l|}
\hline
Concept	&Sec.	&Suggested	&Relevant	\\\hline
Separable Review&4 & 42 &1-8,23-45	\\\hline
Exact &5 & 5,11,26,29,34& 1-13,24-40,56-65	\\\hline
Integrating Factors&5 & 21,22,41,47 & 21,22,41-42,47-49,51,55	\\\hline
Linear&6 &4,13,20,32,51 & 1-6,9-15,20-36,43-49,50-57	\\\hline
Homogeneous&4, &11,12,48 &11-17,46-54	\\\hline
Bernoulli &6 & 16,53& 16,17,37-42,53	\\\hline
Applications&7 &4[27],6[33],1[38] & 1-6 [26-44]	\\
&7 &10[48],17[67],7[88] & 8-10 [45-50],16-18[65-70], 7[87-88]	\\\hline
Laplace Review &21 & 19,32,33[use table] & 4-7,10-12,27-35	\\\hline
Inverse Transforms &22 & 1,2,3,6,13,15 & 1-3,6,15,17,20-28,42,42,45-47	\\\hline
Solving ODEs &24 & 1,14,19(parfrac) & 1,2,11,14,15,19-19,22,24,25,38-42 	\\\hline
\end{tabular}
\end{center}
\section{Basic Concepts and Vocabulary}
Let's start this chapter with a review problem from the first chapter. 


\begin{problem}
 Solve the ordinary differential equation $y'-5y=0$. Then use the initial condition $y(0) = 7$ to obtain the unknown constant.
\end{problem}

\begin{definition}[Differential Equation Language]
A differential equation is an equation which involves derivatives (of any order) of some function.  
\begin{itemize}
 \item An \textbf{ordinary differential equation (ODE)} is a differential equation involving a function $y(x)$ whose domain is one dimensional. The function only has ordinary derivatives.
 \item A \textbf{partial differential equation (PDE)} is a differential equation involving a function $y(x_1, x_2, \ldots)$ whose domain is more than one dimensional. The function has partial derivatives.  
 \item The order of an ODE is the largest order derivative that appears in the ODE. 
 \item A solution to an ODE on an interval $(a,b)$ is a function $y(x)$ defined on the interval $(a,b)$ which satisfies the ODE.  
\end{itemize}
\end{definition}

To verify that a function is a solution to an ODE, calculate derivatives and put them in the ODE. If the resulting equation is an identity for all $x\in (a,b)$, then you have verified that you have a solution. 

Typically a solution to an ODE involves an arbitrary constant $C$. There is often an entire family of curves which satisfy a differential equation, and the constant $C$ just tells us which curve to pick. 
\begin{definition}[Initial Value Problems (IVP)]
Often an ODE comes with an \textbf{initial condition} $y(x_0)=y_0$ for some values $x_0$ and $y_0$. 
\begin{itemize}
 \item A \textbf{general solution} of an ODE is all possible solutions of the ODE.  
 \item A \textbf{particular solution} is one of infinitely many solutions of an ODE. 
 \item We can use the initial conditions to find a particular solution of the ODE. 
 \item An ODE, together with an initial condition, is called an \textbf{initial value problem (IVP)}. 
\end{itemize}
\end{definition}

This next problem has you practice with the vocabulary above.  You'll want to use separation of variables to solve this problem.
\begin{problem}\label{ex2 in chapter 4}
Consider the IVP $y'-4y=8$, where $y(0)=3$.   
\begin{enumerate}
 \item What is the order of the ODE?
 \item Obtain a general solution of the ODE. State an interval on which your general solution is valid.
 \item Verify that your general solution is a solution to the ODE. 
 \item Solve the IVP.
\end{enumerate}
\end{problem}

One of the key uses of differential equations is their ability to model the world around us. If something is changing, then we can often use $y'$ to represent that change.  If we know a force is acting on an object, then $F=ma = my''$ allows us to build a differential equation that models the motion of the object.  As the semester progresses, we'll be making these connections in each chapter, and showing how to use differential equations to model the world.  We'll also see that eigenvalues and eigenvectors are the connecting piece that allows us to see, and obtain, the solution to differential equations. Many of the models we build will depend on observing that a change is proportional to something, or that a force is proportional to something. If you've forgotten what proportional means, here's a definition.

\begin{definition}[Proportional]
 We say that $y$ is proportional to $x$ if $y=kx$ for some constant $k$. We call the constant $k$ the proportionality constant. When two quantities are proportional, then doubling one will double the other, tripling one will triple the other, and so on. A percentage change to one is matched by the other.
\end{definition}

The next problem has us build our first model.
Suppose you go to the doctor's office to get a strep test done.  They swab the back of your throat and then put a sample of tissue from your body in a petri dish.  If you have strep, then the bacteria will grow inside the petri dish, and they'll be able to see the rapid growth of the strep bacteria in a fairly short amount of time. 
\begin{problem}[Exponential Growth]\label{exponential growth application}
Suppose that you place some bacteria in a petri dish. Initially, there are $P$ mg of the bacteria in the dish, and then the bacteria starts to reproduce, so the amount of the bacteria is changing. Let $y(t)$ represent the mg of bacteria in the dish after $t$ days. Then $y'$ would represent the rate at which $y$ is changing. The rate at which $y$ grows depends on how large $y$ is.  If you were to double $y$, then the growth rate $y'$ should double as well. Similarly, if you tripled $y$, then the growth rate $y'$ would triple as well. It seems reasonable to assume that $y'$ is proportional to $y$.
\begin{enumerate}
 \item Express the statement ``$y'$ is proportional to $y$'' as a differential equation. What are the initial values (if any)?
 \item Solve the differential equation above, obtaining a general and particular solution.
 \item Interpret your solution in the context of the original problem. What does a typical graph of your solution look like (it's got some constants in it, but you can show the general shape). If your solution is correct, what will happen as $t$ gets large?
 \item If after 10 minutes you measure 5 mg of the bacteria, and then after 20 minutes you measure 8 mg of the bacteria, how much bacteria was present initially? [If you apply the natural logarithm to both sides of your solution, then you can solve a linear system of equations to obtain the unknowns $\ln P$ and $k$. You can then use Cramer's rule or RREF.]
\end{enumerate}
\end{problem}

The next problem is very similar to the previous, we'll just change the setting from growth of a bacterial culture, to growth of an investment.
\begin{problem}
 Suppose you invest $P=\$10,000$ dollars in an account, and that the money accumulates interest at a constant rate. Let $A(t)$ represent the accumulated worth of your investment after the investment has had $t$ years to grow. No new deposits are made, rather the interest is just left in the account to accumulate more interest.
\begin{enumerate}
 \item Why is is reasonable to assume that $A'$ is proportional to $A$?
 \item Express the connection between $A$ and it's growth as a differential equation. What are the initial values (if any)?
 \item Solve the differential equation, obtaining a general and particular solution.
 \item Interpret your solution in the context of the original problem. What does a typical graph of your solution look like (it's got some constants in it, but you can show the general shape). If your solution is correct, what will happen as $t$ gets large?
 \item Suppose after 5 years that the value of the investment has reached \$18,000.  How long will it take for the investment to reach \$100,000.
\end{enumerate}

\end{problem}


Let's look at one more application before introducing additional solution techniques.  Here's the scenario.
You decide to cook a turkey for Thanksgiving. You turn the oven on to 350$^\circ$F, and the package says that you need to get the turkey heated up to an internal temperature of 165$^\circ$F.  You followed the instructions and thawed the turkey so that currently it's about 40$^\circ$F.  How long will it take for the turkey to heat up? If instead of heating a turkey, you wanted to heat a chicken patty, would the time vary?  If you just wanted to heat a metal pan up, how would the time vary? The next problem introduces a simplistic model to examine this question.  The model works best when you assume that an increase in heat is evenly distributed throughout an object (such as heating a metal pan). When you heat a turkey, the heat is not evenly distributed. This uneven heat distribution complicates the following model, and we'd need to explore PDEs to obtain a better model for heat flow.  To simplify things, we'll assume that heat distributes itself evenly throughout the object. 
\begin{problem}[Newton's Law of Cooling]
 Suppose that you place an object in an oven.  The oven temperature is set to $A$ (you can use Fahrenheit, Celsius, or Kelvin). I'm using $A$ as the temperature of the surrounding ``a''tmosphere. The object's initial temperature is $T_0$.  Let $T(t)$ represent the temperature of the object $t$ minutes after we place the object in the oven. If $T(t)$ is really close to $A$, then the rate at which $T$ increases should be pretty small, as the temperature of the object is almost the same as the temperature of the atmosphere.  If $T$ is really far from $A$, then the rate of temperature change should be a lot larger.  Hence, it appears that $T'$ depends on the difference $A-T$.  Newton conjectured that the rate at which the temperature changes is proportional to the difference $A-T$.
\begin{enumerate}
 \item Express the statement ``the rate at which the temperature changes is proportional to the difference $A-T$'' as a differential equation. What are the initial values (if any)?
 \item Solve the differential equation above, obtaining a general and particular solution.
 \item Interpret your solution in the context of the original problem. What does a typical graph of your solution look like (it's got some constants in it, but you can show the general shape). If your solution is correct, what will happen as $t$ gets large? Does this seem reasonable.
\end{enumerate}
\end{problem}
 

\begin{problem}
 You should have obtained the solution to Newton's law of Cooling as $$T(t) = A+(T_0-A)e^{-kt},$$ where $k$ is the proportionality constant. Suppose that $T_0=45^\circ$F and $A=350^\circ$F.  
\begin{enumerate}
 \item After 5 minutes, you check the temperature and observe $T(5)=80^\circ$F.  What is $k$, and how long will it take for the object to reach $165^\circ$F. 
 \item After 5 minutes, you check the temperature and observe $T(5)=120^\circ$F.  What is $k$, and how long will it take for the object to reach $165^\circ$F. 
 \item The number $k$ depends on the material you are trying to heat.  If $k$ is large, what does that mean about the material?  Think of some examples where $k$ would be large, and where $k$ would be small.
\end{enumerate}
\end{problem}

You've now seen a few examples of how differential equations are used to model the world around us. You will most likely find that in your future courses, you'll be taking real world phenomenon and expressing the relationships you see as differential equations. Solving those differential equations gives us mathematical models we can use to interpret the world around us. There are three parts to this process. 
\begin{itemize}
 \item Express real world phenomenon in terms of a differential equation.
 \item Solve the differential equation.
 \item Interpret the solution in the context of the problem, which often involves using the results to predict behavior.
\end{itemize}
A main focus in this class will be the second portion, ``Solve.''  As we all come from a different background, we won't have time to develop the background material that you'll explore in your respective majors, so the ``Express'' portion will often come in your major courses. You may find in some future courses that they focus on the ``Express'' and ``Interpret'' portions, and then refer you to some standard reference for the ``Solve'' part. The goal of our course is to help you develop the key solution techniques. Along the way, we'll occasionally add some simpler problems that we can ``Express'' and ``Interpret'' without needing a lot of background.

\section{Solution Techniques}
In the review chapter, we explored finding potentials of a gradient field.  We also introduced the language of differential forms.  Recall the following definition. 
\begin{definition}[Differential Forms]
Assume that $f,M,N$ are all functions of two variables $x,y$.
\begin{itemize}
\item A differential form is an expression of the form $Mdx+Ndy$ (just as a vector field is a function  $\vec F=(M,N)$).
\item The differential of a function {$f$} is the expression {$df = f_xdx+f_ydy$} (just as the gradient is $\vec \nabla F = (f_x,f_y)$).
\item \marginpar{A differential form is exact precisely when the corresponding vector field is a gradient field.}%
If a differential form is the differential of a function {$f$}, then we say the differential form is exact (just as we say a vector field is a gradient field). The function {$f$} is called a potential for the differential form. Let me reiterate.  We say a differential form $Mdx+Ndy$ is exact if and only if there exists a function $f$ such that $$df=Mdx+Ndy.$$
\end{itemize} 
\end{definition}


The next problem provides the key idea need to solve almost every differential equation we'll encounter in this course. If you can rewrite the differential equation in differential form, and the differential is exact, then solving the ODE requires that you find a potential.
\begin{problem} 
Consider the differential form $(2x+3y)dx+(3x)dy$. 
\begin{enumerate}
 \item By taking derivatives, show that the differential form is exact. [See the test for a conservative vector fields, problem \ref{test for a conservative field}.] Show that a potential for this differential form is $f(x,y) = x^2+3xy$.  
 \item Rewrite the differential equation $3xy'+3y=-2x$ in the differential form 
$$Mdx+Ndy=0.$$ 
  What's the angle between the vectors $(M,N)$ and $(dx,dy)$?
 \item Explain why the solution to $Mdx+Ndy=0$ is a level curve of the potential $f(x,y)$.
 \item Give the solution to $3xy'+3y=-2x$ if $y(2)=1$. 
\end{enumerate}
\end{problem}

I'm trying to decide on a good name for the next theorem.  We'll see that this theorem is crucial to solving just about EVERY differential equation we encounter from here on out, and it also solves all the ones before now. The name below might change, but something along the lines of ``the sledgehammer,'' or ``one tool to rule them all'' would work. The theorem has no official name, so we can make one up as we go. Basically, we'll show that we can reduce almost every ODE that we solve to a form which allows us to apply the following theorem.
\begin{theorem}[The sledgehammer for ODEs - one tool to rule them all]
 Suppose that $Mdx+Ndy$ is an exact differential form with potential $f(x,y)$.  If we can write an ordinary differential equation in the form $Mdx+Ndy=0$, then an implicit general solution to the ODE is $f(x,y)=c$. The level curves of a potential are precisely the solutions to the ODE. Let me repeat that. The level curves of a potential are precisely the solutions to the ODE.
\end{theorem}

Let's use the previous theorem now to solve a couple of ODEs.
\begin{problem}
 Give a general solution to each of the following ODEs. You may give your solution implicitly, so don't worry about solving for $y$. [Hint: Use the previous theorem.]
\begin{enumerate}
 \item $(4x+2y)dx+(2x+y)dy=0$
 \item $(x\cos(xy)+y)y'=\sin x-y\cos xy$
\end{enumerate}
\end{problem}


\subsection{Use integrating factor when the ODE is not exact}
Let's now return to a problem we've already solved, and show how we can use the sledgehammer theorem to solve things we've already seen, provided we add one more step.
\begin{problem}\label{integrating factor introduction}
 Consider the ODE $y'=-3y$ which we can write in differential as $3ydx+1dy=0.$
\begin{enumerate}
 \item Show that $3ydx+1dy$ is not exact. Then use separation of variables to solve the ODE.
 \item Multiply both sides of  $3ydx+1dy=0$ by $\frac{1}{y}$.  Show that the resulting differential form is exact, and use the sledgehammer theorem to obtain a solution.
 \item Multiply both sides of  $3ydx+1dy=0$ by $e^{3x}$.  Show that the resulting differential form is exact, and use the sledgehammer theorem to obtain a solution.  
\end{enumerate}
\end{problem}

Any time we can write an ODE in the differential form $Mdx+Ndy=0$, the zero on right hand side gives us power.  Our goal will be to multiply both sides of the differential equation by some function $F$, called an integrating factor, so that the resulting differential is exact.  The general solution to the ODE is then simply the level curves of a potential. 
\begin{definition}[Integrating Factor]
 An integrating factor for a differential form $M(x,y)dx+N(x,y)dy$ is a function $F(x,y)$ so that the product $FMdx+FNdy$ is exact.
\end{definition}
In Problem \ref{integrating factor introduction}, I gave you two different integrating factors. Where did they come from?  The next problem will show you how I obtained one of the integrating factors.  There many more options.

\begin{problem}\label{integrating factor that depends only on x}
Let $M(x,y)dx+N(x,y)dy$ be a differential form.  For simplicity, we just write $Mdx+Ndy$.  Suppose $F(x,y)$ is an integrating factor.
\begin{enumerate}
 \item To be exact, explain why we must have 
$$
\dfrac{\partial F}{\partial y}M+F\dfrac{\partial M}{\partial y} 
= 
\dfrac{\partial F}{\partial x}N+F\dfrac{\partial N}{\partial x} 
$$
\item If we assume that $F$ only depends on $x$, so that $F(x,y)=F(x)$, show that a possible option for an integrating factor is
$$F(x)=e^{\int \frac{M_y-N_x}{N} dx} = \exp\left(\int \frac{M_y-N_x}{N} dx\right).$$
\item If we assume that $F$ only depends on $y$, so that $F(x,y)=F(y)$, show that a possible option for an integrating factor is
$$F(y)=e^{\int \frac{N_x-M_y}{M} dy} = \exp\left(\int \frac{N_x-M_y}{M} dy\right).$$
\end{enumerate}
[In class, you may omit the last part in your presentation, as it's almost an exact replica of the 2nd part.] 
 
\end{problem}

The problem above gives us a way of finding integrating factors for many differential equations. It will not give an integrating factor for EVERY differential equation, but it will provide an integrating factor for almost all the ODEs we tackle in this course. Let's now try using this technique on a problem we've already solved.


\begin{problem}
Consider the ODE $y'-4y=8$, which we solved in Problem \ref{ex2 in chapter 4}.  
\begin{enumerate}
 \item Rewrite the ODE in differential form $Mdx+Ndy=0$. 
 \item Find an integrating factor $F(x)=e^{\int \frac{M_y-N_x}{N} dx} = \exp\left(\int \frac{M_y-N_x}{N} dx\right).$
 \item Multiply both sides by the integrating factor, and then solve the ODE by applying the sledgehammer theorem.
\end{enumerate}
\end{problem}

\begin{problem}\label{solving ODEs by finding an integrating factor}
Solve each ODE by finding an appropriate integrating factor. 
\begin{enumerate}
 \item  $y'+4xy = 3x$
 \item  $2y dx+(3x+4y)dy=0$ (Doable now)
 \item  $y'+3y=e^{2x}$ (Solve for $y$.)
 \item  $y'-4y=e^{4x}$ (Solve for $y$.)
 \item  $xy'-4y=2x$ (Solve for $y$.)
\end{enumerate}
\end{problem}

Let's now look at an additional application. 
We will encounter mixing model problems throughout the semester. 
They provide a simple way to see applications of ODEs, without requiring much background. 
\begin{problem} [Mixing Model]
Suppose a 2000 gallon tank contains a solution of water which initially contains 50 lbs of salt. The tank has an inflow valve, and an outflow value.  We would like to change the salt content, so we start pumping in 30 gallons of water (with 1/2 lb of salt per gallon) each minute. We'll assume that the mixture is evenly spread throughout the entire tank by constant stirring.  At the same time, 30 gallons of the evenly stirred mixture flow through the outflow valve each minute. Let $y(t)$ represent the lbs of salt in the tank after $t$ minutes. We currently only know $y(0)=50$. Our goal is to determine the amount of salt $y(t)$ in the tank after $t$ minutes.
\begin{enumerate}
 \item (Express)
 The salt content changes in two ways. Salt is added through the new solution (a flow in), and salt leaves through the outlet valve (flow out).  
 Explain how to obtain a formula for the flow in, and a formula for the flow out. Then explain why
$$y'=15-\frac{30}{2000}y.$$
 \item (Solve) Obtain a general solution to the ODE, and then use the initial value to obtain a particular solution.
 \item (Interpret) Construct a graph of your solution.  As $t$ increases, what happens to the salt content? Does your answer seem reasonable?
\end{enumerate}

\end{problem}

The mixing model problem above, as well as the exponential model and Newton's law of cooling, all belong to a special class of ODEs which we call linear ODEs.

\begin{definition}[Linear ODE]
 If we can write an ODE in the form $y'+p(x)y=q(x)$, then we say that the ODE is linear. This is precisely because the operator $L(y) = y'+a(x)y$ is a linear operator.  If $q(x)=0$, then we say the linear ODE is homogeneous. Otherwise, we say the linear ODE is non homogeneous.
\end{definition}

The next problem provides a way to obtain a solution to EVERY linear ODE. Practice until you can develop this formula quickly, and then you'll have the key concepts needed for solving just about every ODE we encounter throughout the semester.

\begin{problem}[A Linear ODE Solution]
 Consider the linear ODE $y'+p(x)y=q(x)$, where $p$ and $q$ are differentiable functions of $x$ on some interval.  Find an appropriate integrating factor, and then find a potential. Finish by solving for $y$ to show that on this interval, a general solution is $$y(x) = e^{-\int p(x) dx}C+e^{-\int p(x) dx} \int \left(e^{\int p(x)dx} q(x) \right)dx,$$ where $C$ is an arbitrary constant. If the linear ODE is homogeneous, what is a general solution?
\end{problem}


\begin{problem*}[14 and $1/2$:]
 Go back to problem {solving ODEs by finding an integrating factor}, and decide which ODEs are linear.  Then pick one of the ODEs and solve it using the general solution from the previous problem.
\end{problem*}

Let's tackle a couple more application problems.  As you solve them, rather than use the formula above, practice finding an appropriate integrating factor, and then find a potential.

\begin{problem}
Suppose a 50 gallon tank contains a solution of fertilizer which initially contains 10 lbs of fertilizer. We start pumping in 4 gallons per minute, where the concentration of fertilizer is 1/3 lb per gallon. Assume that the mixture is evenly spread throughout the entire tank by constant stirring. The extra  At the same time, 4 gallons of the evenly stirred mixture flow through the outflow valve each minute. Let $y(t)$ represent the lbs of fertilizer in the tank after $t$ minutes.
\begin{enumerate}
 \item Express the mixing model as an IVP (give the ODE and the IV).
 \item Solve the IVP.
 \item Construct a rough graph of your solution.  As $t$ increases, what happens to the salt content? Does your answer seem reasonable?from the 
\end{enumerate}
\end{problem}

The next problem applies Newton's law of cooling to examine what happens if the temperature of the surrounding environment changes. Recall that Newton's law of cooling suggests that the rate of change of temperature of an object is proportional to the difference between the current temperature and the surrounding atmosphere, which we wrote earlier as $$T'=k(A-T).$$
\begin{problem}
 Suppose that during a summer day, the temperature outdoors fluctuates between 70$^\circ$F and 110$^\circ$F.  We'll approximate this with a sine wave. If we let $t=0$ be noon, then we could obtain the temperature $A$ outdoors after $t$ hours using the formula 
$$A(t) = 20\sin(\frac{2\pi}{24} t)+90.$$ Suppose that your air conditioner breaks at noon (your house was at 70$^\circ$F at noon), and then by 6pm in the evening, the temperature had risen to 90$^\circ$F.
\begin{enumerate}
 \item Express this heating problem as an IVP.
 \item Show that the ODE is linear, and then use technology to solve the ODE. You'll need to use $T(6)=90$ to obtain the proportionality constant $k$. 
 \item Graph your solution for 3 days. In the late evenings, which is hotter, the house or the outdoors?  
\end{enumerate}
\end{problem}


\subsection{Use a substitution when you can't get an integrating factor.}
We can solve most of the differential equations we tackle this semester by obtaining an integrating factor using the formulas developed in the previous section.  Sometimes however, this won't work. In these cases, we often just have to make an appropriate change of coordinates (a $u$-substitution). Let's illustrate how this works with an example.  Then we'll tackle the logistics model and introduce another application. 

\begin{problem}
 Consider the ODE $y'=\sin(x+y)$. There is no way that you'll get an integrating factor out of this by using our formulas for $F(x)$ and $F(y)$.  The problem is the $x+y$.  We now do a substitution.
\begin{enumerate}
 \item Write the ODE in the differential form $$\begin{bmatrix}M&N\end{bmatrix}\begin{bmatrix}dx\\dy\end{bmatrix}=0.$$
 \item Let $x=x$ and $u=x+y$ (this is a coordinate transformation). Explain why we have
$$\begin{bmatrix}dx\\dy\end{bmatrix}=\begin{bmatrix}1&0\\-1&1\end{bmatrix}\begin{bmatrix}dx\\du\end{bmatrix}.$$
 \item Show that the ODE can be written in the form $(-\sin u - 1)dx+du=0$.  Then use either separation of variables (or the sledgehammer) to solve the ODE.  Don't forget to substitute back in when you're done.
\end{enumerate}
\end{problem}
  
The last problem introduced the key idea.  If you have an ODE in the form 
$$\begin{bmatrix}M&N\end{bmatrix}\begin{bmatrix}dx\\dy\end{bmatrix}=0,$$
then an appropriate substitution $x=x$, $y=g(x,u)$ will give us the ODE
$$\begin{bmatrix}M&N\end{bmatrix}\begin{bmatrix}1&0\\g_x&g_u\end{bmatrix}\begin{bmatrix}dx\\du\end{bmatrix}=0.$$
If we can find an integrating factor $F(x)$ or $F(u)$ which makes 
$$F\begin{bmatrix}M&N\end{bmatrix}\begin{bmatrix}1&0\\g_x&g_u\end{bmatrix}\begin{bmatrix}dx\\du\end{bmatrix}$$
exact, then we can use the sledge hammer to solve the ODE.  The hard part here is finding the correct transformation.  If you can find the correct transformation, then you can solve the ODE.  This is not easy, and in general it can be really tough. 


\begin{problem}
 Consider the ODE $xy y' = 4x^2+2y^2$.  
\marginpar{Notice that the coefficients $xy$, $4x^2$, and $2y^2$, all are basically second order monomial terms. When the coefficients of the ODE are monomials with the same degree, the substitution $u=y/x$ will convert the ODE into a separable ODE. I'll leave this to you to prove. You do have the tools to prove it.}  
In this situation, if you let $u=y/x$ (so $y=xu$), show that you can rewrite the ODE as 
$$\frac{u}{4+u^2}du = \frac{1}{x}dx.$$  This is a separable ODE, which we can solve.  Solve the ODE.
\end{problem}

\begin{problem}
 Consider the ODE $(x+2y)dx+(3x+4y)dy=0$.  Use the substitution $u=y/x$ to convert this into a separable ODE and give a general solution. 
\end{problem}

 Consider the ODE $y'+3y=4y^3$. This ODE is not linear (why?).  We could separate variables on this ODE and solve (we'll do so in class, reminding you about partial fractions). Instead, Bernoulli noticed that if the ODE is in the form $y'+a(x)y=b(x)y^n$, then the substitution $u = y^{1-n}$ will always convert the ODE into a linear ODE, and then we can use an integrating factor to solve the ODE. It's not easy to discover the right substitution that will convert an ODE into something we can solve.  We call them Bernoulli ODEs because his discovery was quite clever.  The $u=y/x$ substitution above was not clever enough to get a name attached to it.

\begin{problem}[Bernoulli ODE]
 Consider the ODE $y'+3y=4y^3$.  Use the substitution $u=y^{1-3}$ to convert this ODE into a linear ODE, and then solve.  
 [Hint:  You know that $u=y^{-2}$.  Use this to solve for $y$, and then compute $dy = ? du$.  Then just substitute. You'll probably have a really ugly term involving $u^{-3/2}$, so multiply both sides by $u^{3/2}$ and all the ugliness will disappear.]
\end{problem}

We'll come back to Bernoulli ODEs and see some applications of them after we review Laplace transforms.


\section{Laplace Transforms}


Recall that the Laplace transform of a function $y(t)$ defined for $t\geq 0$ is $$Y(s)=\mathscr{L}\{y(t)\}=\int_0^\infty e^{-st}y(t)dt.$$ 
\begin{itemize}
 \item We call the function $y(t)$ the inverse Laplace transform of $F(s)$, and we write $y(t)=\mathscr{L}^{-1}\{Y(s)\}$. 
 \item As a notational convenience, we describe original functions $y(t)$ using a lower case $y$ and input variable $t$ or $x$. 
 We describe transformed functions $Y(s)$ using the same capital letter and input variable $s$.
\end{itemize}
We've computed quite a few Laplace transforms already.  For convenience, I've placed the Laplace transforms we'll use most often in Table \ref{laplacetable}. Feel free to use this table as you find Laplace transforms and their inverses.  With practice, you will memorize this table.
\begin{table}
\begin{center}
\begin{tabular}[t]{|c|cc|}
\hline
$f(t)$ & $F(s)$ & provided\\
\hline\hline
$1$					&$\dfrac{1}{s}$ 							&$s>0$\\\hline
$t$				&$\dfrac{1}{s^{2}}$ 			&$s>0$\\\hline
$t^2$				&$\dfrac{2}{s^{3}}$ 			&$s>0$\\\hline
$t^n$				&$\dfrac{n!}{s^{n+1}}$ 			&$s>0$\\\hline
$e^{at}$		&$\dfrac{1}{s-a}$ 			&$s>a$\\\hline
\end{tabular}
\quad
\begin{tabular}[t]{|c|cc|}
\hline
$f(t)$ & $F(s)$ & provided\\
\hline\hline
$\cos(wt)$  &$\dfrac{s}{s^2+\omega^2}$ 			&$s>0$\\\hline
$\sin(wt)$  &$\dfrac{\omega}{s^2+\omega^2}$ 			&$s>0$\\\hline
$\cosh(wt)$ &$\dfrac{s}{s^2-\omega^2}$ 			&$s>|\omega|$\\\hline
$\sinh(wt)$ &$\dfrac{\omega}{s^2-\omega^2}$ 			&$s>|\omega|$\\\hline
$y$     	&$\mathscr{L}\{y\}=Y$ 					&\\\hline
$y'$		&$\begin{array}{rl}s\mathscr{L}\{y\}-y(0)\\ =sY-y(0)\end{array}$ &\\\hline
\end{tabular}

\caption{Table of Laplace Transforms}
\label{laplacetable}
\end{center}
\end{table}

We can use this table, and the linearity of the Laplace transform, to quickly compute both forward transforms and inverse transforms.  The next problem asks you to do this.
\begin{problem}
 Use the table of Laplace transforms to do the following:
\begin{enumerate}
 \item Compute the Laplace transform of $y(t) = 6+2t+4t^2-5e^{7t}+11\cosh(3t)$. 
 \item Compute the inverse Laplace transform of $$\ds Y(s) = \frac{5}{s}+\frac{4}{s^3}+\frac{3s}{s^2+16}-\frac{2}{s^2-9}.$$ Once you have a guess for the inverse Laplace transform, verify that your guess is correct by computing the Laplace transform (using the table of course).
\end{enumerate}
\end{problem}

\begin{problem}
Find the inverse Laplace transform of $F(s) = \dfrac{2s+1}{s^2+5s+4}$.  [Hint: Use a partial fraction decomposition. Start by factoring the denominator.]
\end{problem}

\begin{problem}
Find the inverse Laplace transform of $F(s) = \dfrac{2s+1}{s^2+9} + \dfrac{5s+7}{s^2-9}$.  [Hint: This can all be done using trig and hyperbolic trig functions.]
\end{problem}



The real power behind the Laplace transform comes from the last formula in the table. 
\begin{theorem}[The Laplace Transform of a Derivative] \label{laplace transform of a derivative}
 Suppose that $y(t)$ is a differentiable function defined on $[0,\infty)$ such that $\ds\lim_{t\to \infty}\frac{y(t)}{e^{st}}=0$ for some $s$. We say that $y(t)$ does not grow faster than some exponential, as the function $e^{st}$ grows faster that $y(t)$ (otherwise the limit would not be zero).  If this is the case, then the Laplace transform of $y'$ is
 $$\mathscr{L}\{y'(t)\} = s \mathscr{L}\{y(t)\} - y(0) = sY - y(0).$$
\end{theorem}

\begin{problem}
 Prove the previous theorem.  In other words, show that $\mathscr{L}\{y'(t)\} = s \mathscr{L}\{y(t)\} - y(0) = sY - y(0).$  [Hint, use integration by parts once, and don't forget to use the bounds.]
\end{problem}

Before illustrating the key value of this theorem, let's fill in the only remaining rules in our Laplace transform table that we have not yet developed. 
\begin{problem}
 In the table of Laplace transforms it also states that the transform of 
$\cos(wt)$ is $\dfrac{s}{s^2+\omega^2}$, and that the transform of $\sin(wt)$  is $\dfrac{\omega}{s^2+\omega^2}$. Pick one of these rules, and then use the definition of the Laplace transform to explain why it is true.  [Hint:  You'll want to use integration by parts twice. See the online text if you want more hints.]
\end{problem}

We'll now use Theorem \ref{laplace transform of a derivative} to solve some ODEs.  You'll see the power behind this method.
\begin{problem}
 Consider the IVP $y'=7y$, $y(0)=A$.  
\begin{enumerate}
 \item Apply the Laplace transform to both sides of this ODE.  You should have an equation involving $Y(s)$. 
 \item Solve for $Y$.  
 \item Now find the inverse transform of $Y$. This is $y(t)$, the solution to the ODE.
 \item We already know how to solve this ODE using either separation of variables, or by finding an integrating factor. Pick one of these methods and obtain a general solution.
\end{enumerate}
\end{problem}
Did you see the process above?  Rather than integrate, we just (1) computed the Laplace transform of both sides, (2) solved an algebraic equation for $Y$, and then (3) obtained the inverse Laplace transform to get $Y$.  

Here's a parable to compare to using Laplace transforms.  You are inside a house that has a single door leading to the downstairs.  You are on the main floor, and need to get downstairs.  The door is locked and you don't know where the key is (you can't figure out how to solve the ODE).  You (1) decide to walk out the front door (you apply the Laplace transform).  Then you (2) walk around the house and find the back door entrance to the basement (you solve for $Y$, and maybe apply a partial fraction decomposition).  Then (3) you walk up to the locked door and unlock it from the other side (you find the inverse transform).

The Laplace transform replaces the problem of integrating with an algebraic problem (often easier to solve). We'll be using it throughout the semester to help us see patterns, and unlock difficult problems.  It works best when the ODE is linear.

\begin{problem}
 Consider the IVP $y'+3y=5$, $y(0)=7$.  (1) Apply the Laplace transform to both sides of this ODE to obtain an equation involving $Y=\mathscr{L}\{y\}$. (2) Solve for the transformed function $Y$. You will need to use a partial fraction decomposition to write $\ds Y = \frac{A}{s+3}+\frac{B}{s}$. (3) Use an inverse transform to obtain the solution $y(t)$.  
\end{problem}

You'll find that with most Laplace transform problems, we'll need a partial fraction decomposition before we can compute an inverse transform. The next problem has you practice the Laplace transform inversion process to solve multiple problems that you know the answer to using simple integration.

\begin{problem}
 For each problem below, use a Laplace transform to solve the ODE. Each problem could be solved with simple integration instead. The point to this problem is to help you see how the Laplace transform gives you, in a different way, information you already know how to obtain.
\begin{enumerate}
 \item $y'=5t^2+7t+3$, $y(0)=C$.
 \item $y'=e^{at}$, $y(0)=C$.
 \item $y'=\cosh(3 t)$, $y(0)=2$.
 \item $y'=\sin(3 t)$, $y(0)=2$.
\end{enumerate}
[Hint:  You'll need a partial fraction decomposition to write $\dfrac{s}{s(s^2+9)} = \dfrac{A}{s}+\dfrac{Bx+C}{s^2+9}$ on part 4.  You'll need a similar idea on 2 and 3.]
\end{problem}

Let's end this section with two more Laplace transform problems, where the initial conditions are not given.

\begin{problem}
 Solve the ODE $y'+4y = e^{3t}$ by using a Laplace transform. No initial condition was given, so you should use something like $y(0)=C$. Because this initial condition involves an arbitrary constant, you may find that Cramer's rule helps you quickly obtain the partial fraction decomposition (rather than row reduction).
\end{problem}

\begin{problem}
 Solve the ODE $y'+3y = \sin(2t)$ by using a Laplace transform. See the previous problem for help about what to do when no initial condition is given. 
\end{problem}


\section{What Method Should I Use?}
In this chapter, we've explored various different techniques to solve first order ODEs. Here's a list.
\begin{itemize}
 \item Separation of variables: The easiest, if you can separate.
 \item Exact: The ODE has a potential. Use the sledgehammer.
 \item Integrating Factors: Make the ODE exact.
 \item Substitution: Change variables so you can make the ODE exact.
 \item Laplace Transforms: Dodge integration. Replace it with algebra.
\end{itemize}
The Laplace transform works nicely on linear ODEs with constant coefficients. If we're missing an initial condition, the algebra gets a little uglier, but still doable. We'll be using the Laplace transform to discover solutions to higher order ODEs as the semester progresses. However, we could have solved every one of the problems we tackled with Laplace transforms by instead using our sledge hammer tool (make the ODE exact, through substitutions and/or integrating factors, and then find a potential). 
\marginpar{I'm working on writing a paper to extend the sledgehammer approach to solve just about every ODE undergraduates tackle, and provide a uniform approach to working with ODEs.   I'm just waiting for an interested student to come and complete the project with me.}
The sledgehammer tool will solve a much larger range of ODEs than the Laplace transform, and near the end of the semester we'll see it's true power in terms of matrices, eigenvalues, and eigenvectors. 

So which method should you use? That depends on how much work you want to do.  The sledgehammer tool will solve EVERY problem we see. If an ODE is separable, it's generally much faster to just use separation of variables (which is really just using an integrating factor).  If the ODE is linear, with constant coefficients, and you have an initial condition, then a Laplace transform might be faster.  If all else fails, make the ODE exact and find a potential.

\begin{problem}
 Which method would you use to solve each ODE below? If you opt for separation of variables, then show us how to separate.  If the ODE is exact, show us how you know.  If you decide to find an integrating factor, show us the integrating factor.  If you will use a substitution, what substitution will you use? If you decide to use Laplace transforms, take the Laplace transform of both sides.  In all cases, don't solve the ODE, rather just show us the first step in the solution process.
\begin{enumerate}
 \item $x^2y'=4xy^2$, $y(2)=1$.
 \item $xy'=3y+x$, $y(2)=1$.
 \item $y'+8y=e^x$, $y(0)=1$.
 \item $y'+8y=y^2$, $y(0)=1$.
\end{enumerate}
 
\end{problem}

Let's end the chapter by considering another application. In Problem \ref{exponential growth application}, we considered the growth of bacteria in a petri dish.  We could have applied this to any other population (such as deer in a forest, people on the Earth, cancer cells spreading through the bloodstream, number of cell phones users in Brazil, speed of computer processors, etc.)  There is a problem with this model.  It works great for a little while, but physical systems cannot grow exponentially forever. Eventually the growth has to slow down.  In the example with the petri dish, eventually the bacteria will have gotten so large that it cannot support more growth. This is where our final problem begins.

\begin{problem}
 Suppose that bacteria grow in a petri dish (if you don't like bacteria, then pick something else to put in here that interests you).  From Problem \ref{exponential growth application}, we used the model $y'=ky$ to express that the rate of growth $y'$ is directly proportional to the size $y$ of the population. We assumed that $k$ was constant.  Here's where we now make a change.  Instead of assuming that $k$ is constant, let's assume that as the population gets larger, that the constant $k$ decreases. In fact, if we let $M$ represent a theoretical maximum population, let's assume that $k$ is proportional to the difference between the current population and this theoretical maximum. 
\begin{enumerate}
 \item (Express) Explain why $y'=-a(y-M)y$. 
 \item (Solve) Solve the ODE using separation of variables.  You'll need to perform a partial fraction decomposition on $\frac{1}{y(y-M)}$.
 \item (Interpret) Pick some constants for $a$, $M$, and the initial size of the population. Then graph your solution. You should obtain a logistics curve (please use a computer to check your work).
\end{enumerate}

\end{problem}

\begin{problem}
 The ODE $y'=-a(y-M)y$ is a Bernoulli ODE.  Rewrite it in the form $y'+a(x)y=b(x)y^n$, and then use Bernoulli's substitution $u=y^{1-n}$ to solve the ODE.  
\end{problem}
 
\begin{question}
 Why can't we (yet) use a Laplace transform to solve $y'=-a(y-M)y$?
\end{question}















