
\noindent This chapter covers the following ideas.

\begin{enumerate}
 \item Explain Hooke's Law in regards to mass-spring systems. Construct and solve differential equations which represent this physical model, with or without the presence of a damper.  
 \item Understand the vocabulary and language of higher order ODEs, such as homogeneous, linear, coefficients, superposition principle, basis, linear independence. 
 \item Solve homogeneous linear ODE's with constant coefficients (with and without Laplace transforms). In addition, create linear homogeneous ODE's given a basis of solutions, or the roots of the characteristic equation.
 \item Explain how the Wronskian can be used to determine if a set of solutions is linear independent.	Briefly mention the existence and uniqueness theorems in relation to linear ODEs, and give a reason for their importance.
\end{enumerate}

The problems below come from Schaum's Outlines \textit{Differential Equations} by Richard Bronson. If you are struggling with a topic from the preparation problem set, please use this list as a guideline to find related practice problems.

\begin{center}
\begin{tabular}{|l|c|l|l|l|l|}
\hline
Concept&Sec&Suggestions&Relevant Problems\\ \hline
Vocabulary of ODEs&8*&33-35&1-3,33-35\\ \hline
2nd Order Homogeneous&9*&1,7,12,21,27,40&1-15, 17-45\\ \hline
nth Order Homogeneous&10*&3,7,8,9,12,18,37,41,44,49&All\\ \hline
IVPs (Homogeneous)&13&9&4,9,13\\ \hline
Applications&14&2,3,5,29,31,34,41-43&1-8,26-43\\ \hline
Laplace Transforms&21*&26, 54&14(c),15(b),25,26,54-58,\\ \hline
Inverse Transforms&22*&7, 34-36,38,read 12 and 18,44&6-10,15-19,29-30,32-53\\ \hline
Solving ODES&24&26,44&5,26,31,36,43,44\\ \hline
Wronskian and Theory&8*&9,10,18,20,43,48,53,58&5-10, 13-20, 31,36-64\\ \hline
\end{tabular}
\end{center}

*The problems in these sections are quick problems. It is important to do lots of them to learn the pattern used to solve ODEs. You may be able to finish 7 or more problems in 15 minutes or less.  Please do more, so that when you encounter these kinds of problems in the future you can immediately give an answer and move forward.


\section{Some Physical Models}
In this chapter, we're going to learn how to solve a huge collection of higher order differential equations.  Before diving into the details, let's make sure we know WHY we would even want to do so. If I knew you all had the same background, we could dive into lots of examples directly related to your field (you'll do that in future classes in your major).  Since we have a diverse background in our class, we'll stick mostly to models that connect velocity, position, and acceleration.  Before the next chapter ends, we'll add to this some information about electrical circuits.


For our first model, let's look at how we can obtain the position of an object in projectile motion from knowledge about the acceleration and velocity.  You've solve this problem before, but the solution required neglecting air resistance. 
\begin{example}
In multivariate calculus, we encountered the differential equation $y''=-g$. In this differential equation, the only force $F_T=my''$ acting on an object in projectile motion is the force of gravity $F_G=-mg$. Equating these two gives us the ODE $my''=-mg$, or just $y''=-g$.  If we have initial position $y(0)=y_0$ and initial speed $y'(0)=v_0$, then the solution is $y=-\frac{1}{2}gt^2+v_0t+y_0$. We found that solution by integrating twice. 
\end{example}
We don't have to neglect air resistance anymore.  We could talk about sky diving (risky), dropping bombs (deadly), throwing math books off a roof (illegal), putting a satellite into geosynchronous orbit (useful), or dropping a pebble from the top of a waterfall (head to Yellowstone and try it - sounds like we need a field trip). The next problem asks you to revisit the example above, but now add in air resistance.
\begin{problem}
 Joe hikes up to the top of Lower Falls in Yellowstone.  His hope is to gauge the height $h$ of the waterfall.  He plans to drop a pebble from the top, and time how long it takes for the pebble to hit the ground. He'll need a model that predicts the height of the pebble at any time $t$.

 For this to work, Joe has to make some assumptions.  His assumptions might be way off, but that's how science works. We start with assumptions and then turn those assumptions into differential equations. Here's what Joe assumes:
\begin{itemize}
 \item He assumes Newton's second law of motion, namely that $F=ma$ (the total force is the mass times the acceleration).
 \item He assumes that the total force is comprised of two parts.  
 \item The first force $F_G$ comes from a constant acceleration due to gravity. He assumes that gravity is constant $a=-g$. The negative sign comes because the acceleration causes a decrease in height.
 \item The second part comes from air resistance. He assumes that the faster the pebble goes, the greater this force will be. If the pebble's speed were to double, then this force should double.  So he assumes that the force due to air resistance $F_R$ is proportional to the pebble's velocity.
\end{itemize}
Let $y(t)$ represent the height, above ground, of the pebble after $t$ seconds. Use Joe's assumptions to answer the following:
\begin{enumerate}
 \item Rewrite Newton's second law of motion in terms of $y$, $y'$, and/or $y''$. 
 \item What is the constant force $F_G$ due to gravity?
 \item Rewrite Joe's assumption about air resistance in terms of $y$, $y'$ and/or $y''$. 
 \item The total force $F$ is the sum of the two forces, i.e. we can write $F = F_G+F_R$. Use this fact, together with your answers from the previous two parts, to obtain a second order ODE.  You don't have to solve the ODE, rather you just need to obtain it.
\end{enumerate}
If you need any hints, try searching the web for ``modeling motion if we assume that air resistance is proportional to speed.''
\end{problem}

Congrats.  You've just set up your first second order ODE. 

Let's now look at another position/velocity/acceleration model, but this time related to springs. 
We'll start by considering the following scenario. We attach an object with mass $m$ to a spring.  
\marginpar{In the next chapter, we'll hang the spring from a ceiling. In this case, we'll have an additional force $F_g=-mg$ acting on the spring.}
We place the spring horizontally, and put the mass on a frictionless track. We let go of the object, and allow it to come to rest. We'll use the function $x(t)$ to keep track of the position of the spring at any time $t$, with $x=0$ corresponding to equilibrium (the mass is at rest). Robert Hooke (1635 -- 1703) developed the following law, called Hooke's law:
\begin{quote}
 The force needed to extend (or compress) a spring a distance $x$ is proportional to the distance $x$. Note that the force acts opposite the displacement.
\end{quote}
\begin{problem}
 Read the preceding paragraph.  Then answer the following:
\begin{itemize}
 \item Draw a picture of a horizontal track. On the left end of the track, put a wall. Put a on object, like a square block, in the center of your track and draw a spring that connects the wall to the block.
 \item Explain why $mx''(t)=-kx(t)$. We generally just write $mx''=-kx$ (the $t$ is assumed).  
 \item If it takes $8 \text{ N} = 8$kg m/s$^2$ to move the object whose mass is 4 kg about $.3$ m, what is the spring constant $k$?  How far would a $12$ N force cause the object to move? Does the mass of the object matter?
\end{itemize}
\end{problem}

Hooke's law is not a perfect model for all springs, but it does a good job for most, provided the displacement is not too large.  If the displacements are too large, then the spring may deform, which changes the properties of the spring in all future computations.  If you take your car out onto extremely bumpy roads, and purposefully hit some nasty bumps, you could permanently damage the shocks. In this case, you would want to replace your springs.

Every linear spring has a spring constant $k$. This constant has many names, such as the spring modulus, Young's modulus, Young's constant, and more. The next problem shows you how to obtain the spring constant $k$.
\begin{problem}
 You attach a spring to the ceiling. You attach a mass of 10 kg on the end, and the spring elongates 3 cm.  
\begin{enumerate}
 \item You now attach a mass of 20 kg. How long will the spring elongate? 
 \item What is the spring constant $k$? Give the units.
 \item We attach a different spring, and hang the same 10 kg on the end, but this time the spring elongates 2 cm.  Is the spring constant larger or smaller?
 \item If a spring has really large modulus, will it be easy or hard to elongate it?
\end{enumerate}
\end{problem}

We need one more model before we start solving some ODEs.  We'll use the exact same spring model as before. Place a horizontal spring whose modulus is $k$ on a frictionless track. Attach an object whose mass is $m$ to the end of the spring.  
\marginpar{We don't have to place the spring underwater to get the same affect.  We could use a dashpot to resist the motion. One type of dashpot is a cylindrical tube placed around a cylindrical object, so that as the object moves, it's sides come in contact with the dashpot, resulting in friction that resists motion. See \href{http://en.wikipedia.org/wiki/Dashpot}{Wikipedia} for more info.}
We now place the entire mass-spring system underwater. When it was exposed to air, we neglected air resistance. Now we'll have to take resistance into account.   
\begin{problem}
 When we have no resistance, the mass-spring system ODE is $F_T=F_S$, or $mx'' = -km$.  Assume that the liquid applies a resistive force that is proportional to the velocity of the object.  If the object is resting, the liquid doesn't apply a force.  If you double the speed, then the resistive force doubles.  If you triple the speed, the resistive force triples. Modify the ODE $mx''=-km$ to account for the resistive force of water. 
\end{problem}

\section{Notation, Vocabulary, and Solutions}

We can write the ODEs from the previous section as
$$my''+ky'=-mg,
\quad mx''+ky=0,
\quad \text{and}\quad mx''+cy'+ky=0.$$
If we divide each ODE by $m$, then we can write each ODE in the general form 
$$y''+p(t) y'+q(t)y=r(t).$$
This introduces our next definition.
\begin{definition}[Linear, Constant Coefficient, and Homogeneous]
\
\begin{itemize}
 \item If we can write an ODE in the form  $y''+p(t)y'+q(t)y=r(t)$, then we say the ODE is a second order linear ODE. 
 \item The functions $p(t)$ and $q(t)$ we call the coefficients of the linear ODE.
 \item If the coefficients are constant, the we say the ODE is a constant coefficient linear ODE.   
 \item If the right hand side $r(t)=0$, then we say the linear ODE is homogeneous. Otherwise we say it is non homogeneous.
 \item We use the words $n$th order linear ODE to talk about any ODE that we can write in the form $y^{(n)}+a_{n-1}(t)y^{(n-1)} + \cdots + a_1(t)y'+a_0(t)y=r(t)$, where $y^{(n)}$ is the $n$th derivative of $y$. 
\end{itemize}
\end{definition}


We just introduces a few new words, so with each problem that follows, let's practice using those words. The next problem has you explain why we use the word ``linear.''

\begin{problem}
 Consider the second order ODE $y''+7y'+6y=0$.  
\begin{itemize}
 \item Why is this ODE linear?  Modify it so it is no longer linear, and show us in class what would make it non linear.
 \item Is this ODE homogeneous?  Explain.
 \item Let $L(y) = y''+7y'+6y$.  Show that $L$ is a linear operator. (See the end of chapter 3 if you need to reread the definition). 
 \item The solutions to the ODE are the solutions to $L(y)=0$.  In the language of linear operators, what do we call the set of functions $y$ such that $L(y)=0$?  It was another key word near the end of chapter 3.  Please look it up. The set of solutions $y$ is the \rule{1in}{.5pt} of $L$.
\end{itemize}
\end{problem}

To solve second order linear homogeneous ODE, we'll use the Laplace transform.  In the previous chapter, we showed that 
$$\mathscr{L}(y') = s\mathscr{L}(y)-y(0) = sY-y(0).$$
We need a rule for second derivatives. Repeated application of the single derivative rule will give you all the rules you need. 
\begin{problem}
 Show that under suitable conditions, we can compute the Laplace transform of the second derivative of $y$ by using the formula
 $$\mathscr{L}(y'')=s^2\mathscr{L}(y)-sy(0)-y'(0) = s^2Y-sy(0)-y'(0).$$
 Then show that 
 $$\mathscr{L}(y''')=s^3\mathscr{L}(y)-s^2y(0)-sy'(0)-y''(0).$$
 Conjecture a formula for the Laplace transform of the 7th derivative of $y$.
 [Hint: As stated in the paragraph before this problem, apply the rule $\mathscr{L}(y') = s\mathscr{L}(y)-y(0)$ multiple times.]
\end{problem}

We are now ready to solve a second order ODE with Laplace transforms.
\begin{problem}
 Consider the IVP $y''+3y'+2y=0$, $y(0)=7$, $y'(0)=5$. 
\begin{enumerate}
 \item Is the ODE linear? Is it homogeneous? Are the coefficients constant?
 \item Compute the Laplace transform of both sides and solve for $\mathscr{L}(y) = Y$.
 \item Use a partial fraction decomposition to show that $$Y=\frac{A}{s+1}+\frac{B}{s+2},$$
where you give the constants $A$ and $B$.
 \item Find the solution $y$ to this IVP by computing the inverse Laplace transform of $Y$.
 \item How are solutions to $s^2+3s+2=0$ connected to your solution?  
\end{enumerate}
\end{problem}

\begin{problem}
 Consider the IVP $y''+7y'+10y=0$, $y(0)=c$, $y'(0)=d$. 
 \begin{enumerate}
 \item Is the ODE linear? Is it homogeneous? Are the coefficients constant?
 \item Compute the Laplace transform of both sides and solve for $\mathscr{L}(y) = Y$.
 \item If we use a partial fraction decomposition, we would write $$Y=\frac{A}{s+2}+\frac{B}{s+5}.$$\
 Why is the solution $y(t)$ a linear combination of $e^{-2t}$ and $e^{-5t}$, i.e. $y(t)=Ae^{-2t}+Be^{-5t}$?
 \item Now actually perform the partial fraction decomposition to obtain the constants $A$ and $B$. (Since you have variables $a$ and $b$ in your system, you'll want to use Cramer's rule).
 \item How are solutions to $s^2+7s+10=0$ connected to your solution?  
\end{enumerate}
\end{problem}

In the previous two problems, we had initial conditions.  When the initial conditions are numbers, it made the partial fraction decomposition rather simple.  When the initial conditions are variables, finding the constants in the partial fraction decomposition was a little trickier. The next problem has you work through a problem when we have no initial conditions.

\begin{problem}
 Consider the ODE $y''+7y'+12y=0$. We would like a general solution (no initial conditions are given).
 \begin{enumerate}
 \item Compute the Laplace transform of both sides and solve for $\mathscr{L}(y) = Y$. You'll have $y(0)$ and $y'(0)$ in the numerator of your solution. It would be nice if they weren't there.
 \item Factor the denominator of $Y$, and write your solution as $Y = \frac{A}{?}+\frac{B}{?}$. This time DO NOT solve for $A$ and $B$. You don't need to.
 \item Compute the inverse Laplace transform of $Y$.  Your answer should involve the unknown constants $A$ and $B$. You've found the general solution.
 \item The polynomial $s^2+7s+12$ showed up in your work above. How are the zeros of this polynomial connected to the solution?
\end{enumerate}
\end{problem}

In the three examples above, we took an ODE $y''+ay'+by=0$, applied a Laplace transform, and obtained the polynomial $s^2+as+b$.   The zeros of this polynomial seem to be intimately connected to the solution. Let's give this polynomial a name.
\begin{definition}[Characteristic Polynomial (Equation)]
 Consider the ODE $y''+ay'+by=0$. 
\begin{itemize}
 \item The characteristic polynomial is $s^2+as+b$. We could alternately use $\lambda^2+a\lambda +b$.
 \item The characteristic equation is $s^2+as+b=0$. We could alternately use $\lambda^2+a\lambda +b=0$.
\end{itemize}
\end{definition}
With this new word, we now have the correct tool to discuss solving ODEs. We noticed a pattern in the first few problems.  From that pattern, we developed a new word.  Now we can use that word to simplify your solution techniques.

\begin{problem}
 Consider the ODE $y''+9y'+20y=0$.  What is the characteristic equation of the ODE?  Find the zeros of the characteristic polynomial, and then state a general solution to the ODE.
\end{problem}

The definition of characteristic equation allows us to alternately use the variable $\lambda$ instead of $s$.  The next problem connects what we are doing to eigenvalues. 
\begin{problem}
 Consider the ODE $y''+9y'+20y=0$ from the previous problem.  If we let $y_1=y$ and $y_2=y'$, then we can write the ODE in the form $y_2'+9y_2+20y_1=0$.  This becomes the system of ODEs
\begin{align*}
 y_1'&=y_2\\
 y_2'&=-20y_1-9y_2.
\end{align*}
 Write the system above in the matrix form $\pvec{y_1\\y_2}' = A \pvec{y_1\\y_2}$.  Then find the eigenvalues of $A$, and use them to obtain a solution to the ODE. 
\end{problem}



Let's now tackle a problem where the characteristic equation does not have real zeros. 
\begin{problem}
 Consider the ODE $y''+16y=0$. 
 \begin{enumerate}
 \item Compute the Laplace transform of both sides of the ODE and solve for $\mathscr{L}(y) = Y$. You'll have $y(0)$ and $y'(0)$ in the numerator of your solution. 
 \item Compute the inverse Laplace transform of $Y$. Your answer will involve $y(0)$ and $y'(0)$. 
 \item What is the characteristic polynomial, and what are its roots?
 \item If a mass of $1$ kg is attached to spring with modulus $16$ kg/s$^2$ on a frictionless track, then graph the position $x(t)$ at any time $t$. [What's the corresponding ODE? Didn't you already solve this ODE?]
\end{enumerate} 
\end{problem}

The previous problem showed us how to tackle a problem where the roots of the characteristic polynomial are purely imaginary. What do we do if the roots repeat, or if they are complex of the form $a\pm bi$?  The next problem addresses this.  

\begin{problem}
 Consider the ODE $y''+6y'+9y=0$.  
\begin{enumerate}
 \item What are the zeros of the characteristic equation? From these zeros, guess a general solution. (It's OK if you're wrong.)
 \item Compute the Laplace transform of both sides of ODE. Then solve for $Y$ and show that 
$$Y = \frac{A(s+3)+B}{(s+3)^2} =  \frac{A}{(s+3)}+\frac{B}{(s+3)^2}. $$
 \item Compute the inverse Laplace transform of each part that you are able to compute, and explain why we can't perform the inverse Laplace transform of the other parts. 
 \item Use a computer to complete the inverse Laplace transform, and state the solution.
\end{enumerate}
\end{problem}

\begin{problem}
 Consider the ODE $y''+4y'+13y=0$.  
\begin{enumerate}
 \item What are the zeros of the characteristic equation?
 \item Compute the Laplace transform of both sides of ODE. Then solve for $Y$ and complete the square to show that 
$$Y = \frac{A(s+2)+B}{(s+2)^2+3^2} = \frac{A(s+2)}{(s+2)^2+3^2} +  \frac{B}{(s+2)^2+3^2}. $$
 \item Use the fact that $\ds \mathscr{L}\{e^{at}\cos(bt)\} = \frac{s-a}{(s-a)^2+b^2}$ and that $\ds \mathscr{L}\{e^{at}\sin(bt)\} = \frac{b}{(s-a)^2+b^2}$ to finish solving. [The next problem will show you where these came from.] 
\end{enumerate}
\end{problem}

In both of the preceding problems, we encountered expressions that we could not inverse transform. The first was $\ds\frac{1}{(s+3)^2}$, and the last two were $\ds\frac{(s+2)}{(s+2)^2+3^2}$ and $\ds\frac{1}{(s+2)^2+3^2}$. In all cases, these look like shifted versions of functions for which we know the inverse Laplace transform.  For example, we know $\ds \mathscr{L}\{\cos 3t\} = \frac{s}{s^2+3^9}$.  The expression $\ds\frac{(s+2)}{(s+2)^2+3^2}$ resembles the expression $\ds\frac{(s)}{(s)^2+3^2}$, rather we just replaced $s$ with $s+2$, which is the same as shifting $s$ left 2. We were told that $\ds \mathscr{L}^{-1}\left\{\frac{s-a}{(s-a)^2+b^2}\right\} = e^{at}\cos(bt)$. What we need is a Laplace transform rule that would allow us deal with $s$ shifting. If we know how to invert $Y(s)$, how do we invert $Y(s-a)$?

\begin{problem}[The $s$-shifting Theorem]
 In this problem you'll develop a rule for the inverse transform of $Y(s-a)$. 
\begin{enumerate}
 \item  We know that $Y(s) = \mathscr{L}\{y(t)\} = \int_0^\infty e^{-st}[f(t)]dt$.  Replace $s$ with $s-a$ and obtain a formula
$$Y(s-a) = \int_0^\infty e^{-st}[?] dt.$$ This gives you a formula $\mathscr{L}\{?\} = Y(s-a)$.
 \item What is the inverse Laplace transform of $1/s^2$?  What is the inverse Laplace transform of $1/(s-4)^2$? What is the inverse Laplace transform of $1/(s+5)^2$?
 \item What is the forward Laplace transform of $\cos(bt)$? What is the forward Laplace transform of $e^{at} \cos (bt)$? What is the forward Laplace transform of $e^{7t}t^3$ and $e^{-7t}t^3$?
\end{enumerate}
[Hint: The $s$-shifting theorem is now in Table \ref{laplacetable2}.  Try to tackle this problem without referring to the table.]
\end{problem}

\begin{table}
\begin{center}
\begin{tabular}[t]{|c|cc|}
\hline
$y(t)$ & $Y(s)$ & provided\\
\hline\hline
$1$					&$\dfrac{1}{s}$ 							&$s>0$\\\hline
$t$				&$\dfrac{1}{s^{2}}$ 			&$s>0$\\\hline
$t^n$				&$\dfrac{n!}{s^{n+1}}$ 			&$s>0$\\\hline
$e^{at}$		&$\dfrac{1}{s-a}$ 			&$s>a$\\\hline
$y'$					&$sY-y(0)$ 						&\\\hline
$y''$					&$s^2Y-sy(0)-y'(0)$ 						&\\\hline
\end{tabular}
\quad
\begin{tabular}[t]{|c|cc|}
\hline
$y(t)$ & $Y(s)$ & provided\\
\hline\hline
$\cos(\omega t)$  &$\dfrac{s}{s^2+\omega^2}$ 			&$s>0$\\\hline
$\sin(\omega t)$  &$\dfrac{\omega}{s^2+\omega^2}$ 			&$s>0$\\\hline
$\cosh(\omega t)$ &$\dfrac{s}{s^2-\omega^2}$ 			&$s>|\omega|$\\\hline
$\sinh(\omega t)$ &$\dfrac{\omega}{s^2-\omega^2}$ 			&$s>|\omega|$\\\hline
$y(t)$  &$Y(s)$ 						&\\\hline
$e^{at}y(t)$  &$Y(s-a)$ 						&\\\hline
\end{tabular}

\caption{Table of Laplace Transforms}
\label{laplacetable2}
\end{center}
\end{table}


To apply the $s$-shifting theorem, we'll need to become good at completing the square.  
If we know the transform is $\ds\frac{2}{s^2+4}$, then the inverse transform is $\sin(2t)$. 
If we know the transform is $\ds\frac{2}{(s+3)^2+4}$, then the inverse transform is $e^{-3t}\sin(2t)$. 
However, we would normally have a characteristic polynomial in the form $s^2+6s+13$, rather than the form $(s+3)^2+4$. 
Once we complete the square, we can apply the $s$-shifting theorem.

\begin{problem}
Complete each of the following:
\begin{enumerate}
 \item Consider the ODE $y''+2y'+5y=0$. Find the characteristic polynomial, complete the square, and state a general solution.
 \item Consider the ODE $y''+6y'+9y=0$. Find the characteristic polynomial, complete the square, and state a general solution.  
 \item Consider the ODE $y''+4y'+3y=0$. Find the characteristic polynomial, complete the square, and state a general solution.  
\end{enumerate}
\end{problem}

Before we get to far, let's practice the $s$ shifting theorem for Laplace transforms.
\begin{problem*}[5.16 and 1/2]
 Complete the following:
\begin{enumerate}
 \item Find the Laplace transform of the following:
\begin{enumerate}
 \item $t^3$ and $ t^3 e^{4t}$
 \item $\cos(2t)$ and $e^{-3t}\cos(2t)$
 \item $3\sin(7t)$ and $3e^{-5t}\sin(7t)$ 
\end{enumerate}
 \item Find the inverse Laplace transform of the following:
\begin{enumerate}
 \item $\dfrac{3}{s^4}$ and $\dfrac{3}{(s-5)^4}$
 \item $\dfrac{s+3}{(s+3)^2+4}$ and $\dfrac{1}{(s+3)^2+4}$
 \item $\dfrac{s}{(s+3)^2+4}$.
\end{enumerate}
\end{enumerate}

\end{problem*}


\begin{problem}
 Consider the ODE $y''+6y'+11y=0$.  
 \begin{enumerate}
  \item Find the characteristic polynomial, complete the square, and then state a general solution.
  \item Find the characteristic equation, use the quadratic formula to solve the characteristic equation, and then state a general solution.
  \item Solve the ODE $y''+5y'+12y=0$. Would you rather complete the square, or use the quadratic formula?
 \end{enumerate}
\end{problem}

\begin{problem}
 Consider the ODE $ay''+by'+cy=0$. 
\begin{enumerate}
 \item Obtain the characteristic equation. Complete the square.  State the zeros of the characteristic equation. [When you finish this problem, you will have proved the quadratic formula.]
 \item If we let $y_1=y$ and $y_2=y'$, we obtain the system of ODE $y_1'=y_2$ and $ay_2'+by_2+cy_1=0$.  
 Write this system in the  matrix form $\pvec{y_1\\y_2}' = A \pvec{y_1\\y_2}$, and obtain the eigenvalues of $A$.  
\end{enumerate}
\end{problem}

We can now solve EVERY second order homogeneous constant coefficient ODE.  All we have to do is find the characteristic equation. The zeros unlock a general solution of the ODE.
\begin{problem}
 Consider the second order homogeneous constant coefficient ODE $y''+ay'+by=0$.  Let $\lambda_1$ and $\lambda_2$ be the roots of the characteristic polynomial $s^2+as+b$.  
\begin{enumerate}
 \item If the roots are real and $\lambda_1 \neq \lambda_2$, then $y(t) = $\rule{1in}{.5pt}.
 \item If the roots are real and $\lambda_1  =   \lambda_2$, then $y(t) = $\rule{1in}{.5pt}.
 \item If the roots are complex where $\lambda = c\pm di$, then $y(t) = $\rule{1in}{.5pt}. \\
 If $c=0$, then the solution is simply $y(t) = $\rule{1in}{.5pt}.
\end{enumerate}

\end{problem}


\begin{problem*}[5.19 Improved]
 Suppose that we have a second order ODE, and we have already computed the roots of the characteristic polynomial to be $\lambda_1$ and $\lambda_2$. 
\begin{enumerate}
 \item If $\lambda_1=-3$ and $\lambda_2=-5$, then $y(t) = $\rule{1in}{.5pt}.\\
 If the roots are real and $\lambda_1 \neq \lambda_2$, then $y(t) = $\rule{1in}{.5pt}.
 \item If $\lambda_1=-3$ and $\lambda_2=-3$, then $y(t) = $\rule{1in}{.5pt}\\
 If the roots are real and $\lambda_1  =   \lambda_2$, then $y(t) = $\rule{1in}{.5pt}.
 \item If $\lambda_1 = -2+3i$ and $\lambda_2=-2-3i$, then $y(t) = $\rule{1in}{.5pt}. \\
 If the roots are complex where $\lambda = a\pm bi$, then $y(t) = $\rule{1in}{.5pt}.
 \item If $\lambda_1 = 5i$ and $\lambda_2=-5i$, then $y(t) = $\rule{1in}{.5pt}. \\
 If the roots are purely imaginary so that $\lambda = bi$, then $y(t) = $\rule{1in}{.5pt}.
\end{enumerate}

\end{problem*}

Have you noticed that every general solution above is a linear combination of two independent solutions?  Recall that we say a differential operator is linear if $L(y_1+y_2) = L(y_1)+L(y_2)$ and $L(cy_1)=cL(y_1)$ for functions $y_1$ and coefficients $c$.  
\begin{problem}[Superposition Principle]
 Suppose that $y_1$ and $y_2$ are both solutions to a linear differential equation $ay''+b y'+cy=0$.  Consider the linear operator $L(y) = ay''+by'+cy$.  Prove that any linear combination of $y_1$ and $y_2$ is also a solution to the ODE $L(y)=0$. (Hint:  Look at the last few problems in chapter 3, or just prove this directly.)  

 Many people refer to this fact as the superposition principle. To get a solution to a second order homogeneous ODE, all you need is two independent solutions. The general solution is any linear combination of them.
\end{problem}

Now that we have a general solution, let's show how to quickly obtain the solution to an IVP. The key principle, is to first obtain a general solution. Differentiate your general solution, and then use your initial conditions to find the unknown constants.

\begin{problem}
 Consider the IVP $y''+6y'+5y=0$, with $y(0)=4$ and $y'(0)=5$.  Obtain a general solution. Then compute $y'(t)$. Plug the initial conditions into both $y$ and $y'$ to solve for the unknown constants in your general solution. 
\end{problem}

\begin{problem}
 Consider the IVP $y''+6y'+9y=0$, with $y(0)=4$ and $y'(0)=5$.  Obtain a general solution. Then compute $y'(t)$. Use the initial conditions to solve for the unknown constants in your general solution. 
\end{problem}

\begin{problem}
 Consider the IVP $y''+2y'+5y=0$, with $y(0)=4$ and $y'(0)=5$.  Obtain a general solution. Then compute $y'(t)$. Use the initial conditions to solve for the unknown constants in your general solution. 
\end{problem}


\section{Mass-Spring Systems}
 Recall from the introductory examples that we can model the position of a spring using the ODE $$mx''+cx'+kx=0$$
 The constants $m$, $c$, and $k$ are physical constants related to the mass-spring system.
\begin{itemize}
 \item The mass of the object attached to the spring is $m$.
 \item The spring constant, or modulus, is $k$.
 \item The coefficient of friction of any attached dashpot is $c$. If no dashpot is attached, then we just let $c=0$.
\end{itemize}

\begin{problem}
 Suppose we attach a mass of $4$ kg to a spring with modulus $12$ kg/s$^2$. We displace the object $1$ cm from the equilibrium position of the spring, and then hit the mass with a hammer. The impact causes the spring's initial velocity to be 3 cm/s back towards equilibrium.  Use this information to determine the position of the spring at any time $t$. Construct a graph of the position. From your graph, show how you can the initial position and initial velocity.
\end{problem}

Make sure you ask me in class to show you how the solution above graphically changes, if we alter the initial position and initial velocity.

\begin{problem}
 Suppose we attach a mass of $m$ kg to a spring with modulus $k$ kg/s$^2$. We displace the object $y_0$ cm from the equilibrium position of the spring, and give the object an initial velocity of $v_0$ cm/s away from equilibrium. In the absence of friction, the mass-spring system will oscillate in a regular pattern. Determine the position of the spring at any time $t$.  What is the period of oscillation? If you doubled the spring constant $k$, how would it affect the period?
\end{problem}

\begin{problem}
 Suppose we attach a mass of $m$ kg to a spring with modulus $k$ kg/s$^2$. We displace the object $y_0$ cm from the equilibrium position of the spring, and give the object an initial velocity of $v_0$ cm/s away from equilibrium. In the absence of friction, the mass-spring system will oscillate in a regular pattern. Give a formula for the amplitude of the oscillation. [Hint: If you write your solution in the form $y(t) = C\sin(\omega t+\phi)$, then you can quickly read off the amplitude. How do you write $y(t) = A\cos(bt)+B\sin(bt)$ in the form $C\sin(\omega t+\phi)$?] 
\end{problem}

Each of the problems above dealt with undamped motion, there was no friction to slow down the motion.  The remaining problems include a dashpot, something placed around the mass-spring system that adds friction to the system. Wikipedia has some excellent pictures of what a dashpot could look like.  I like to think of an old screen door, and the cylindrical tube at the bottom of the door that helps close the door and prevent it from smashing closed on little fingers.  Ask me in class to show you a dashpot on our classroom door.

\begin{problem}
 Recall from the introductory examples that we can model the position of a spring using the ODE $mx''+cx'+kx=0$. 
 We now attach a mass of 1 kg to a spring. The spring is placed inside a dashpot, to add friction to the system, and the dashpot has a coefficient of friction equal to $c=8$ kg/s. The spring is rather large, so we extend it 1 m and then release it with no initial velocity. 
\begin{enumerate}
 \item If the spring modulus is $k=15$ kg/s$^2$, find the position $x(t)$ of the spring, and construct a rough sketch of $x$ versus $t$.  
 \item If the spring modulus is $k=16$ kg/s$^2$, find the position $x(t)$ of the spring, and construct a rough sketch of $x$ versus $t$.  
 \item If the spring modulus is $k=17$ kg/s$^2$, find the position $x(t)$ of the spring, and construct a rough sketch of $x$ versus $t$.  
 \item What connection is there between $c$ and $k$? If you had to describe what you saw in the examples above to someone not in this class, what would you say?  You'll probably have to explain this phenomenon to a boss someday. 
\end{enumerate}
\end{problem}



\section{Higher Order ODEs}
In the previous sections, we focused mainly on second order ODEs.  We started by using Laplace transforms to find the exact solutions.  The accompanying partial fraction decomposition was sometimes rather ugly, so we opted for guessing the form of the solution, and then taking derivatives to determine the unknown constants. 

\begin{problem}
 Consider the ODE $y'''+3y''+3y'+y=0$.  Compute the Laplace transform of both sides.  The characteristic equation is $(s+1)^3=0$. Explain why the solution is $y = c_1 e^{-x}+c_2 x e^{-x}+c_3 x^2e^{-x}.$ 

 For the ODE $y''''+4y'''+6y''+4y'+y=0$, whose characteristic equation is $(s+1)^4=0$, make a guess as to the solution. Then use a computer and dsolve to check that your answer is correct. (Wolfram Alpha can solve this.)

 If you encounter a repeated root, what does that contribute to the solution? Explain this in a way that you and other can remember it.
\end{problem}

\begin{problem}
 You have a 6th order homogeneous ODE, and the characteristic equation factors as $(s^2+4)(s^2+9)^2=0$. What is the original ODE (expand the polynomial)? The roots are $\pm 2i, \pm 3i, \pm 3i$ (so $\pm 3i$ are repeated roots). Guess the general solution.  Then use a computer to check if your guess was correct.
\end{problem}

\begin{problem}
 In each problem below, you'll be given the characteristic equation of an ODE. State the general solution of the ODE.
\begin{enumerate}
 \item $(s+3)(s+2)(s+1)=0$
 \item $(s+3)(s+3)(s+1)=0$
 \item $(s+3)^3(s^2+9)=0$
 \item $(s+3)^2(s^2+9)^2=0$
 \item $(s+3)^2(s^2-9)^2=0$ (Note the sign change)
\end{enumerate}
\end{problem}


\section{Existence and Uniqueness - the Wronskian}


This section will be added to soon.
