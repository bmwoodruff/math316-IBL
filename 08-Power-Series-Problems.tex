

\newgeometry{left=1in,right=1in,top=1in,bottom=1in}


\section{Extra Practice Problems}

Extra homework for this unit is right here.
Make sure you try a few of each type of problem, ASAP. I suggest that the first night you try one of each type of problem. It's OK if you get stuck and don't know what to do, as long as you decide to learn how to do it and then return to the ones where you got stuck. Eventually do enough of each type to master the ideas.  The only section in Schaum's with relevant problems is chapter 27.  Handwritten solutions are available online.  \href{https://content.byui.edu/file/664390b8-e9cc-43a4-9f3c-70362f8b9735/1/08-Power-Series-Preparation-Solutions.pdf}{Click for solutions.}

Most engineering textbooks assume you have seen Taylor series and power series before (in math 113), but many of you have not. If you have your old Math 215 book, you can find many relevant problems and explanations in the section on the Ratio Test and Taylor Series.  

Here are a few key functions and their Taylor series centered at $x=0$ (their MacLaurin series).
\begin{center}
\begin{tabular}{|c|c|c||c|c|c|}\hline
$f(x)$ & MacLaurin Series & Radius&$f(x)$ & MacLaurin Series & Radius\\\hline
$\ds e^x$ & $ \ds\sum_{n=0}^\infty \frac{1}{n!}x^n$& $R=\infty$&
$\ds \frac{1}{1-x} $&$   \ds\sum_{n=0}^\infty x^n$& $R=1$\\\hline
$\ds \cos(x) $&$  \ds \sum_{n=0}^\infty \frac{(-1)^n}{(2n)!}x^{2n}$& $R=\infty$&
$\ds \cosh(x) $&$   \ds\sum_{n=0}^\infty \frac{1}{(2n)!}x^{2n}$& $R=\infty$\\\hline
$\ds \sin(x) $&$   \ds\sum_{n=0}^\infty \frac{(-1)^n}{(2n+1)!}x^{2n+1}$& $R=\infty$&
$\ds \sinh(x)$&$  \ds\sum_{n=0}^\infty \frac{1}{(2n+1)!}x^{2n+1}$& $R=\infty$\\\hline
\end{tabular} 
\end{center}




\begin{multicols}{2}

\begin{enumerate}

\item [(I)] For each of the following, find a Taylor polynomial of degree $n$ centered at $x=c$ of the function $f(x)$.
\begin{multicols}{2}
\item $e^{4x}, n=3, c=0$
\item $\cos(x), n=4, c=\pi$
\item $\cos(2x), n=4, c=0$
\item $\sin(\frac12 x), n=5, c=0$
\item $\frac{1}{x}, n=3, c=1$
\item $\ln x, n=3, c=1$
\item $\ln (1-x), n=4, c=0$
\item $\ln (1+x), n=4, c=0$
\end{multicols}






\item [(II)] Find the radius of convergence of each power series.
\begin{multicols}{2}
\item $\ds \sum_{n=0}^\infty \frac{1}{3^n}x^n$
\item $\ds \sum_{n=0}^\infty \frac{(-1)^n}{4^{n+1}}x^{n}$
\item $\ds \sum_{n=0}^\infty \frac{n}{2^n}x^{3n}$
\item $\ds \sum_{n=0}^\infty \frac{3n+1}{n^2+4}x^n$
\item $\ds \sum_{n=0}^\infty \frac{(-4)^n n}{n^2+1}x^{2n}$
\item $\ds \sum_{n=0}^\infty \frac{n}{2^n}x^{2n}$
\item $\ds \sum_{n=0}^\infty \frac{(-1)^n}{n!}x^{n}$
\item $\ds \sum_{n=0}^\infty \frac{n!}{10^n}x^{2n}$
\end{multicols}





\item [(III)] For each function, find the MacLaurin series and state the radius of convergence.
\begin{multicols}{2}
\item $f(x)=e^x$
\item $f(x)=\cos x$
\item $f(x)=\sin x$
\item $\ds f(x)=\frac{1}{1-x}$
\item $\ds f(x)=\frac{1}{1+x}$
\item $f(x)=\cosh x$
\item $f(x)=\sinh x$
\end{multicols}






\item [(IV)] Prove the following formulas are true by considering power series. These formulas will allow us to eliminate complex numbers in future sections.
\item $e^{ix}=\cos x + i\sin x$ (called Euler's formula)
\begin{multicols}{2}
\item $\cosh(ix) = \cos x$
\item $\cos(ix) = \cosh x$
\item $\sinh(ix) = i\sin x$
\item $\sin(ix) = i\sinh x$
\end{multicols}







\item [(V)] Use MacLaurin series of known functions to find the MacLaurin series of these functions (by integrating, differentiating, composing, or multiplying together two power series). Then state the radius of convergence.
\item $f(x)={x^2}{e^{3x}}$
\item $f(x)=\frac{x^2}{e^{3x}}$ [hint, use negative exponents]
\item $f(x)=\cos 4x$
\item $f(x)=x\sin(2x)$ 
\item $f(x)=\frac{x}{1+x}$
\item $f(x)=\frac{1}{1+x^2}$
\item $f(x)=\arctan x$ [hint, integrate the previous]
\item $f(x)=\arctan (3x)$ 





\item [(VI)] Shift the indices on each sum so that it begins at $n=0$.
\begin{multicols}{2}
\item $\ds\sum_{n=3}^6 n+2$ 
\item $\ds\sum_{n=2}^8 n^2$ 
\item $\ds\sum_{n=4}^\infty 2^n$ 
\item $\ds\sum_{n=2}^\infty x^n$ 
\item $\ds\sum_{n=1}^\infty n a_n x^{n}$ 
\item $\ds\sum_{n=2}^\infty n (n-1) a_n x^{n-2}$ 
\end{multicols}






\item [(VII)] Solve the following ODEs by the power series method. With some, initial conditions are given (meaning you know $y(0)=a_0$ and $y^\prime(0)=a_1$). Identify the function whose MacLaurin series equals the power series you obtain.
\item $y^{\prime}=3y$
\item $y^{\prime}=2xy$
\item $y^{\prime\prime}+4y=0$
\item $y^{\prime\prime}-9y=0, y(0)=2, y^\prime(0)=3$
\item $y^{\prime\prime}+4y^{\prime}+3y=0, y(0)=1, y^\prime(0)=-1$



\item [(VII)] Determine whether the given values of $x$ are ordinary points or singular points of the given ODE.
\item Chapter 27, problems 26-34 (these are really quick).


\item [(VIII)] Solve the following ODEs by the power series method.  State the recurrence relation used to generate the terms of your solution, and write out the first 5 nonzero terms of your solution.
\item Chapter 27, problems 35-47 (or from the worked problems).






\end{enumerate}
\end{multicols}

\section{Extra Practice Solutions}
Handwritten solutions are available online.  \href{https://content.byui.edu/file/664390b8-e9cc-43a4-9f3c-70362f8b9735/1/08-Power-Series-Preparation-Solutions.pdf}{Click for solutions.}

\section{Special Functions}
Here are some extra practice problems related to the Frobenius method and other special functions. Section numbers correspond to problems from Schaum's Outlines \textit{Differential Equations} by Richard Bronson. The suggested problems are a minimum set of problems to attempt. 

\begin{center}
\begin{tabular}{|l|l|}
\hline
Concept	&Relevant Problems\\\hline
Frobenius Method*&28:1-4, 5-10, 12,14,16,18-20\\ \hline
Legendre Polynomials&27:11-13; 29:4,6,8,11,12,15 \\ \hline
Bessel Functions&30:9,11,12,26, 27,\\ \hline
Gamma Functions& 30:1-8, 24, 25\\ \hline
Substitutions&28:22-23, 34-38; 30:30,31 \\ \hline
\end{tabular}
\end{center}

\href{https://content.byui.edu/file/664390b8-e9cc-43a4-9f3c-70362f8b9735/1/09-Special-Functions-Preparation-Solutions.pdf}{Click here for some handwritten solutions to many of the problems above. }

\restoregeometry


