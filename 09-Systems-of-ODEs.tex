
This chapter covers the following ideas. When you create your lesson plan, it should contain examples which illustrate these key ideas. Before you take the quiz on this unit, meet with another student out of class and teach each other from the examples on your lesson plan. 


\begin{enumerate}
\item Explain the basic theory of systems of linear ODEs and the Wronskian for systems.
\item Convert higher order ODEs to first order linear systems.
\item Explain how to use eigenvalues and eigenvectors to diagonalize matrices. When not possible, use generalized eigenvectors to find Jordan canonical form.
\item Find the matrix exponential of a square matrix, and use it to solve linear homogeneous and nonhomogeneous ODEs. 
\item Give applications of systems of ODEs. In particular be able to setup systems of ODE related to dilution, electricity, and springs (use the computer to solve complex systems). 
\end{enumerate}



\section{Bringing it all together}
As you work on the problems in this section, you'll want to have a computer algebra system near by.  I'll put some links to Sage worksheets in the problem set, but I strongly suggest you download the Mathematica Technology Introduction.

Our goal in this chapter is to learn how to solve systems of differential equations.  We have already discussed most of the ideas in this chapter (in some context), but we have never brought all these ideas together.  In this chapter, we'll try to connect everything we have done up to now. By the time we end this chapter, we'll have a tool that will solve almost every problem we have encountered. We'll see how vector fields, parametric curves, eigenvalues, eigenvectors, potentials, and power series all combine together to give a beautiful and elegant solution technique to solving ODEs.

\begin{problem}
 Consider the IVP $y''+3y'+2y=0$, $y(0)=5$, $y'(0)=0$. This solution to this ODE will give the position of a mass spring system where $m=1$ kg, $c=3$ kg/s, $k=2$ kg/s$^2$, where the object was lifted upwards 5 cm and then let loose. 
\begin{enumerate}
\item This is a homogeneous ODE. What is the characteristic equation? State a general solution, and then use the initial conditions to get the solution.  (This is review.)
\item Let $y(t)$ be the position and $v(t)$ be the velocity, so $v(t)=y'(t)$.  This means $v'(t)+3v(t)+2y(t)=0$.  Notice that this is a first order ODE with two different functions $y$ and $v$ that are unknown. Explain how to write this as the matrix equation
$$
\begin{bmatrix}
y'\\v'
\end{bmatrix}
=\begin{bmatrix}
0&1\\-2&-3
\end{bmatrix}
\begin{bmatrix}
y\\v
\end{bmatrix}.
$$
 \item Find the eigenvalues $\lambda_1$ and $\lambda_2$ of the coefficient matrix 
$A=
\begin{bmatrix}
0&1\\-2&-3
\end{bmatrix}
.$ 
Then for each eigenvalue, find a corresponding eigenvector, which we'll call $\vec x_1$ and $\vec x_2$. 
\item Look back at your solution on part 1. Compute $y'$ and write your solution from part 1 in the form 
$$
\begin{bmatrix}
y\\y'
\end{bmatrix}
=
\begin{bmatrix}
?\\?
\end{bmatrix}
e^{-t}+
\begin{bmatrix}
?\\?
\end{bmatrix}
e^{-2t}
.$$
What is the connection between the solution to the ODE and eigenvalues/eigenvectors?  
\item Suppose you needed to find a general solution to the system of ODEs
$$
\begin{bmatrix}
y'\\v'
\end{bmatrix}
=A
\begin{bmatrix}
y\\v
\end{bmatrix}
,$$ and you knew the eigenvalues $\lambda_1$ and $\lambda_2$ and a corresponding eigenvector for each. Make a guess for what a general solution to the ODE is.  
\end{enumerate}
\end{problem}

\begin{problem}
Consider again the IVP $y''+3y'+2y=0$, $y(0)=5$, $y'(0)=0$.  We already have the solution above. In this problem, you'll be constructing various graphs. 
\begin{enumerate}
\item \marginpar{Please check your answer with technology.  You can use either Sage or Mathematica.  \href{http://bmw.byuimath.com/dokuwiki/doku.php?id=systems_of_odes_grapher}{Click on this link to get some example code that will help you with this problem.} You should use this code to check your answer with the previous problem.}%
Construct of graph of $y$ verses $t$ and a graph of $v$ versus $t$. You should have two graphs that show you position and velocity at any time $t$.  
\item Now construct a graph of $v$ versus $y$.  We could call this a velocity-position graph. Please use technology to do this.  You just need to graph the parametric curve $\vec r(t) = (y(t), y'(t))$. You'll need to use a parametric plotter. 
\item The matrix $A$ represents a vector field $\vec F(y,v) = (0y+v, -2y-3v)$. Construct a graph of this vector field in the $yv$ plane.
\item Put your vector field plot, and your velocity-position plot, on the same set of axes.  What does the vector field plot tell you about the velocity-position plot?
\item \marginpar{Just change the initial conditions in either Sage worksheet or Mathematica notebook, and reevaluate.}%
Change the initial conditions to $y(0)=0$ and $v(0)=5$.  On top of your vector field plot, draw what you think the solution should look like in the velocity-position plot. Then use software to solve the ODE, and plot your solution.
\item Change the initial conditions to $y(0)=5$ and $v(0)=-5$.  On top of your vector field plot, draw what you think the solution should look like in the velocity-position plot. Explain why the solution must follow a straight line in the velocity-position plane?  [Hint: What are the eigenvalues, eigenvectors?]  Then state another set of initial conditions where the solution will be a straight line towards the origin.
%\item We know the eigenvalues of $A$ are $\lambda_1 = -1$ and $\lambda_2=-2$, with corresponding eigenvectors $(1,-1)$ and $(1,-2)$. How can you see these in the  
\end{enumerate}

\end{problem}

We'll revisit the last two problems as part of every other solution find.  Let's introduce a new type of application that shows us the need for linear systems of ODEs.

\begin{problem}
 Imagine for a moment that you have two tanks. The first tank contains 6 lbs of salt in 10 gallons of water. The second tank contains no salt in 20 gallons of water.  Each tank has an inlet valve, and an outlet value.  We attach hoses to the tanks, and have a pump transfer 2 gallon/minute of solution from tank 1 to tank 2, and vice versa from tank 2 to tank 1. So as time elapses, there are always 10 gallons in tank 1 and 20 gallons in tank 2. Our goal is to find the amount of salt in each tank at any time $t$. 
\begin{enumerate}
 \item We know there are initially 6 lbs of salt in tank 1, and no salt in tank 2. If we allow the pumps to transfer salt for enough time, explain why the salt content in tank 1 will drop to 2 lb, and the salt content in tank 2 should increase to 4 lbs.
 \item Let $y_1(t)$ and $y_2(t)$ be the lbs of salt in tanks 1 and 2, respectively.  Explain why 
$$y_1 ' = -\frac{2}{10}y_1+\frac{2}{20}y_2.$$
 \item Obtain a similar equation for $y_2'$. Write your ODEs in the matrix form 
$$
\begin{pmatrix}
 y_1'\\y_2'
\end{pmatrix}
=
\begin{bmatrix}
 -2/10 & 2/20\\
 ? & ?
\end{bmatrix}
\begin{pmatrix}
 y_1\\y_2
\end{pmatrix}
$$
\item Draw the vector field represented by the coefficient matrix. Sketch the solution ($y_1(t),y_2(t)$) to your IVP (start at the point $(6,0)$ and follow the field until the vectors no longer tell you to move).  Show that you should stop at $(2,4)$.
\item \marginpar{Don't forget that you can check your work with technology.  \href{http://bmw.byuimath.com/dokuwiki/doku.php?id=systems_of_odes_grapher}{Please following this link} }%
Compute the eigenvalues and eigenvectors of the matrix $A$, and draw two lines through the origin to represent the eigenvector directions.
\end{enumerate}

\end{problem}


\begin{problem}
 Again consider the mixing tank problem from before, with the system of ODEs
$$
\begin{pmatrix}
 y_1'\\y_2'
\end{pmatrix}
=
\begin{bmatrix}
 -2/10 & 2/20\\
 ? & ?
\end{bmatrix}
\begin{pmatrix}
 y_1\\y_2
\end{pmatrix}.
$$
Our goal is to determine the amount of salt in each tank at any time $t$. 
\begin{enumerate}
 \item Compute the eigenvalues and eigenvectors of this matrix.  Use them to write a general solution to this system of ODEs. Your solution should involve arbitrary constants $c_1$ and $c_2$.
 \item Use the initial conditions $y_1(0)=6$ and $y_2(0)=0$ to solve for $c_1$ and $c_2$.
 \item \marginpar{Don't forget that you can check your work with technology.  \href{http://bmw.byuimath.com/dokuwiki/doku.php?id=systems_of_odes_grapher}{Please following this link} }%
Construct a graph that contains the the vector field representing the coefficient matrix and the parametric plot $(y_1(t),y_2(t))$ of your solution.
\end{enumerate}

\end{problem}


\begin{problem}
 Two tanks are connected with hoses and pumps so that 3 gallons/second flows back and forth between the tanks.  The first tank is a 60 gallon tank, with 2 lbs of salt inside.  The second tank is a 90 gallon tank with 23 lbs of salt in it. Please find the amount of salt in each tank at any time $t$. 
\begin{enumerate}
 \item Write a linear system of ODEs in the form 
$$
\begin{pmatrix}
 y_1'\\y_2'
\end{pmatrix}
=
A
\begin{pmatrix}
 y_1\\y_2
\end{pmatrix}.
$$
whose solution will give the amount of salt in each tank at any time $t$.
 \item Compute the eigenvalues and eigenvectors of A, and then write a general solution to this system of ODEs. Your solution should involve arbitrary constants $c_1$ and $c_2$.
 \item Use the initial conditions to solve for $c_1$ and $c_2$.
 \item \marginpar{Don't forget that you can check your work with technology.  \href{http://bmw.byuimath.com/dokuwiki/doku.php?id=systems_of_odes_grapher}{Please following this link} }%
Construct a graph that contains the the vector field representing the coefficient matrix and the parametric plot $\vec y(t) = (y_1(t),y_2(t))$ of your solution.
\end{enumerate}

\end{problem}


\begin{problem}
 Consider the linear system of ODEs given by $y_1'=2y_1+y_2$ and $y_2'=3y_1+4y_2$. Let $\vec y = (y_1,y_2)$. We can write this ODE in the form $\vec y' = A\vec y$, where $\vec y = (y_1,y_2)$.  
 \begin{enumerate}
  \item Find the eigenvalues and eigenvectors of the coefficient matrix $A$.
  \item We know that we can write a general solution to this system of ODEs as 
$$
\vec y=
c_1\vec x_1 e^{\lambda_1 t }+c_2\vec x_2 e^{\lambda_2 t}.
$$
Find a 2 by 2 matrix $Q$ so that we can write this solution in the form 
%\marginpar{Hint: Expand the product of the matrices. You'll see what $Q$ must be.}
$$
\vec y=
Q\begin{bmatrix}
 e^{\lambda_1 t} & 0\\
 0 & e^{\lambda_2 t}
\end{bmatrix}
\begin{bmatrix}
c_1\\c_2 
\end{bmatrix}
=QD\vec c,
$$
where we have the diagonal matrix $D=
\begin{bmatrix}
 e^{\lambda_1 t} & 0\\
 0 & e^{\lambda_2 t}
\end{bmatrix}$ 
and the vector 
$\vec c =
\begin{bmatrix}
c_1\\c_2 
\end{bmatrix}
$.
\item We now have $\vec y = QJ\vec c$. When we let $t=0$, explain why $D$ equals the identity matrix. This means that $\vec y(0) = Q\vec c$.  Using an inverse matrix, we have $\vec c= Q^{-1}\vec y(0)$.  Compute the inverse of $\vec Q$.
\item 
Since we know $\vec y = QJ\vec c$ and $\vec c= Q^{-1}\vec y(0)$, this means 
$$\vec y = QDQ^{-1}\vec y(0).$$
You have found $Q$, $D$, and $Q^{-1}$. If we let $y_1(0)=a$ and $y_2(0)=b$ which means $\vec y(0)=(a,b)$, then multiply out the matrix product $QDQ^{-1}\vec y(0)$ and state the solution to this IVP.
\end{enumerate}

\end{problem}

Do you notice that in the problem above, we solved the linear system of ODEs in the form $\vec y' = A\vec y$  with initial conditions $\vec y(0)$ by just writing
$$\vec y = QDQ^{-1}\vec y(0).$$
The columns of $Q$ were the eigenvectors.  The nonzero entries of the diagonal matrix $D$ contain $e^{\lambda t}$ where $\lambda$ is an eigenvalue.  Does this pattern work in other places?

\begin{problem}
 Solve the system of ODEs $\vec y '=A\vec y$ where 
$A
=\begin{bmatrix}
  6&2\\2&3
 \end{bmatrix}
$. Do so by stating $Q$, $D$, and $Q^{-1}$, and then perform the matrix product $QDQ^{-1}$. Finally, if we assume $\vec y(0) = (a,b)$, then give the solution to this system of IVPs by stating what $y_1(t)$ equals, and what $y_2(t)$ equals (hint, multiply out $QDQ^{-1}\vec y(0)$  ).  Please use technology to perform as much of the computations as you want. Just be prepared to tell us how you got each part.
\end{problem}

Does the pattern above continue to work if we increase the size of the matrix?
\begin{problem}
 Solve the system of ODEs $\vec y '=A\vec y$ where 
$A
=\begin{bmatrix}
  2&1&1\\1&2&0\\0&0&4
 \end{bmatrix}
$. Do so by stating $Q$, $D$, and $Q^{-1}$, and then perform the matrix product $QDQ^{-1}$. Finally, if we assume $\vec y(0) = (a,b,c)$, then give the solution to this system of IVPs by stating what $y_1(t)$ equals, what $y_2(t)$ equals, and what $y_3(t)$ equals. Please use technology to perform as much of the computations as you want. Just be prepared to tell us how you got each part.
\end{problem}

Does the pattern even work if the eigenvalues are complex?
\begin{problem}
 Solve the system of ODEs $\vec y '=A\vec y$ where 
$A
=\begin{bmatrix}
  0&1\\-1&0
 \end{bmatrix}
$. 
\begin{enumerate}
 \item Find the eigenvalues by hand.  Then for each eigenvalue, compute an eigenvector by hand. State $Q$ and $D$ from this information.  You are welcome to use $e^{it}$ in your work as needed.
 \item Compute $Q^{-1}$ by hand.  Then compute the product $QDQ^{-1}$ by hand.
 \item Use Euler's formula $e^{it} = \cos t +i\sin t$ to simplify the product $QDQ^{-1}$. You should be able to simplify the product to remove all complex terms.
 \item If $y_1(0)=5$ and $y_2(0)=7$, then what are $y_1(t)$ and $y_2(t)$. Then use software to check your answer. You should see that the solution in the $y_1y_2$ plane, along with the appropriate vector field, gives circular motion.
\end{enumerate}

\end{problem}

\begin{problem}
Suppose that $\frac{d\vec y}{dt} = A\vec y$ is a linear system of ODEs.  Also suppose that $\vec y=\vec x e^{ct}$ is a nonzero solution to this system. Explain, using the definition of eigenvalues and eigenvectors, why we must have that $c$ is an eigenvalue, and $\vec x$ is an eigenvector corresponding to $c$. [Hint:  Look up the definition of eigenvalues and eigenvectors.  If you compute $\frac{d\vec y}{dt}$ and then place both $\vec y$ and $\frac{d\vec y}{dt}$ into the system $\frac{d\vec y}{dt}= A\vec y$, you should see the definition appear.] 
\end{problem}


\section{The Matrix Exponential}
In the previous section, we saw that if $\vec y' = A\vec y$, then the solution is $\vec y = QDQ^{-1}\vec c$, where the initial conditions give us $\vec c = \vec y(0)$ because $D$ is the identity matrix when $t=0$.  In the first week of class, we solved the differential equation $y'=ay$, and obtained the solution $y=e^{at}c$ where $c=y(0)$.  In this section, we'll show that if we replace the constant $a$ with a matrix of constants $A$, then the solution is still $\vec y=e^{At}\vec c$. To do this, we have to go back to power series. 

\begin{definition}[The Matrix Exponential]
 We showed in the power series chapter that 
$$e^x = 1+x+\frac{1}{2!}x^2+\frac{1}{3!}x^3+\frac{1}{4!}x^4+\frac{1}{5!}x^5+\cdots.$$
We define the matrix exponential of $A$ to be the series
$$e^A = I+A+\frac{1}{2!}A^2+\frac{1}{3!}A^3+\frac{1}{4!}A^4+\frac{1}{5!}A^5+\cdots.$$
The matrix $I$ is the identity matrix.
\end{definition}


\begin{problem}
 Use the definition above to complete the following:
\begin{enumerate}
 \item We know that $e^0=1$. If $A=O=\begin{bmatrix}0&0\\0&0\end{bmatrix}$, then compute $e^{A}$.
 \item If $A=I=\begin{bmatrix}1&0\\0&1\end{bmatrix}$, show that $e^{A}=\begin{bmatrix}e^1&0\\0&e^1\end{bmatrix}$.
 \item If $A=\begin{bmatrix}2&0&0\\0&3&0\\0&0&5\end{bmatrix}$, then compute $e^A$. Make sure you show how you get your answer from the definition.
 \item If $At=\begin{bmatrix}2t&0&0\\0&3t&0\\0&0&5t\end{bmatrix}$, then compute $e^{At}$. You are welcome to just state an answer here. 
\end{enumerate}
\end{problem}

When a matrix is diagonal, it's matrix exponential is simple to compute.  Our main goal is to learn how to compute the matrix exponential of all matrices. Let's look at another type of matrix where it's easy to compute the matrix exponential. 

\begin{problem}[Nilpotent Matrices]
 Use the definition of the matrix exponential to compute the following.
\marginpar{We say that a matrix $A$ is nilpotent if $A^n$ is the zero matrix for some $n$. It's easy to compute the matrix exponential of a nilpotent matrix, because the infinite series stops, and then we just have to add up finitely many terms.}
\begin{enumerate}
 \item Let $At=\begin{bmatrix}0&t\\0&0\end{bmatrix}$ and then compute $(At)^2$ and $(At)^3$.  Use this to state the matrix exponential of $At$. 
 \item Let $At=\begin{bmatrix}0&t&0\\0&0&t\\0&0&0\end{bmatrix}$ and then compute $(At)^2$ and $(At)^3$.  Use this to state the matrix exponential of $At$. 
 \item Let $At=\begin{bmatrix}0&t&0&0\\0&0&t&0\\0&0&0&t\\0&0&0&0\end{bmatrix}$. Give the matrix exponential of $At$. You are welcome to guess your answer by following any pattern you saw above. 
 \item Let $At=\begin{bmatrix}0&t&0&0&0\\0&0&t&0&0\\0&0&0&t&0\\0&0&0&0&t\\0&0&0&0&0\end{bmatrix}$. Guess the matrix exponential of $At$. \marginpar{You can check your answer with technology. \href{http://bmw.byuimath.com/dokuwiki/doku.php?id=matrix_exponential_calculator}{Follow this link.}}
\end{enumerate}
\end{problem}

With real numbers, we have the exponential rule $e^{a+b}=e^a\cdot e^b$.  The exponential of a sum is the same as the product of an exponential. Does this rule work with matrices as well?  Let's try it and see.

\begin{problem}
 Let's write 
$At=
\begin{bmatrix}
2t & t\\
0 & 2t
\end{bmatrix}
=\begin{bmatrix}
2t & 0\\
0 & 2t
\end{bmatrix}
+
\begin{bmatrix}
0 & t\\
0 & 0
\end{bmatrix}
 = Bt+Ct$, 
where 
$Bt=
\begin{bmatrix}
2t & 0\\
0 & 2t
\end{bmatrix}
$ and 
$Ct=
\begin{bmatrix}
0 & t\\
0 & 0
\end{bmatrix}$
.
\begin{enumerate}
 \item \marginpar{\href{http://bmw.byuimath.com/dokuwiki/doku.php?id=matrix_exponential_calculator}{Follow this link. Please use this calculator to check your answers on the other parts of this problem, but only after you first do them by hand.}}%
Use software to compute the matrix exponential of $At$. 
 \item State the matrix exponentials of both $Bt$ and $Ct$ (use the patterns developed from the previous problems). We know that $At=Bt+Ct$, so how should we combine $e^{Bt}$ and $e^{Ct}$ to get the matrix exponential of $At$?
 \item Without software, state the matrix exponential of 
$
\begin{bmatrix}
3t& t &0&0\\
0 & 3t&t&0\\
0 & 0 &3t&t\\
0 & 0 &0&3t
\end{bmatrix}.
$
\end{enumerate}

\end{problem}


\begin{problem}
Consider the matrix 
$At = 
\begin{bmatrix}
 2t&t&0&0&0\\
 0&2t&0&0&0\\
 0&0&2t&t&0\\
 0&0&0&2t&t\\
 0&0&0&0&2t
\end{bmatrix}
$. There is supposed to be a zero instead of a $t$ in the second row.  That was done on purpose. 
\begin{enumerate}
 \item Write $At$ as the sum $At=Bt+Ct$, where $Bt$ is a diagonal matrix, and $Ct$ contains nonzero terms above the diagonal. 
 \item Compute the matrix exponential of both $Bt$ and $Ct$. Then compute their product to get $e^{At}$.  Check your answer with software.
 \item \marginpar{Check your answer with software.}%
Guess the matrix exponential of 
$
\begin{bmatrix}
 3t& t& 0& 0& 0& 0& 0& 0& 0& 0 \\
 0 &3t& t& 0& 0& 0& 0& 0& 0& 0 \\
 0 & 0&3t& t& 0& 0& 0& 0& 0& 0 \\
 0 & 0& 0&3t& 0& 0& 0& 0& 0& 0 \\
 0 & 0& 0& 0&3t& t& 0& 0& 0& 0 \\
 0 & 0& 0& 0& 0&3t& 0& 0& 0& 0 \\
 0 & 0& 0& 0& 0& 0&4t& 0& 0& 0 \\
 0 & 0& 0& 0& 0& 0& 0&4t& t& 0 \\
 0 & 0& 0& 0& 0& 0& 0& 0&4t& t \\
 0 & 0& 0& 0& 0& 0& 0& 0& 0&4t  
\end{bmatrix}
$
\end{enumerate}

\end{problem}




We can now compute the matrix exponential of any matrix that is either diagonal, or has nonzero entries above the diagonal.  We'll soon see that this means we can compute the matrix exponential of every matrix.  The key is to first find the correct form.

\begin{problem}
Consider the matrix 
$A=
\begin{bmatrix}
 2 & 4 \\
 4 & 2 
\end{bmatrix}.
$
\begin{enumerate}
 \item Find the eigenvalues of $A$.  For each eigenvalue, find a corresponding eigenvector.
 \item Let $Q$ be a matrix whose columns are the eigenvectors from the previous part. Compute $A Q$ by hand. 
 \item \marginpar{If you're struggling with guessing what $J$ should be, then compute $Q^{-1}AQ$.  Then try to explain why $J$ should be what you see from the equation $AQ=QJ$. }%
If $\vec x$ is an eigenvector, then by definition we must have $A\vec x = \lambda x$.  
 Use this to write $AQ=QJ$ where $J$ is a diagonal matrix. What is the matrix $J$? Show that $Q^{-1}AQ =J$, and that $QJQ^{-1}=A$.
 \item \marginpar{Did you notice on this last part that you could construct $A$ solely from knowledge about eigenvalues and eigenvectors.}
Now suppose that $A$ is a matrix with eigenvalues $2$ and $3$ and corresponding eigenvectors $(1,3)$ and $(-1,2)$. Use this to state $Q$, $J$, and $A$.
\end{enumerate}
\end{problem}

\begin{problem}
 Suppose we know that $A = QJQ^{-1}$ where $J$ is a diagonal matrix.  In this problem,  we'll compute the matrix exponential of $A$.
\begin{enumerate}
 \item Explain why $A^k = QJ^kQ^{-1}$.\note{They are completely unable to do this.  I need to build more scaffolding.}
 \item Use the definition of the matrix exponential to explain why $e^{A} = Qe^{J} Q^{-1}$.
 \item The matrix $A = \begin{bmatrix}2&1\\1&2\end{bmatrix}$ has an eigenvalue $3$ with corresponding eigenvector $(1,1)$, and has an eigenvalue $1$ with corresponding eigenvector $(1,-1)$.  Compute the matrix exponential of $A$. 
 \item Solve the system of ODEs $y_1' = 2y_1+y_2$ and $y_2' = y_1+2y_2$. State $y_1(t)$. 
\end{enumerate}
\end{problem}

If we know how to compute the matrix exponential of $J$ and $A=QJQ^{-1}$, then the previous problem showed us that 
$\exp (A) = Q \exp(J) Q^{-1}.$ 
We also saw that multiplication by $t$ doesn't affect this result, so we have 
$$\exp (At) = Q \exp(Jt) Q^{-1}.$$ 
This is the key tool we'll use to solve systems of ODEs. 

\begin{problem}
 Consider the system of ODEs $\dfrac{d\vec y}{dt} = A\vec y$ given by  
$$
\begin{pmatrix}
y_1\\
y_2\\ 
y_3
\end{pmatrix}'
=
\begin{bmatrix}
 2&1&0\\
 1&2&0\\
 0&0&3\\
\end{bmatrix}
\begin{pmatrix}
y_1\\
y_2\\ 
y_3
\end{pmatrix}.
$$
\begin{enumerate}
 \item For the coefficient matrix above, state $Q$ and $J$. You should have a repeated eigenvalue, but you should also find two linearly independent eigenvectors corresponding to this eigenvalue. 
 \item Compute $e^{At}$, and state a general solution $\vec y(t)$. 
 \item If $y_1(0)=1, y_2(0)=-2, y_3(0)=2$, then state $y_1(t)$, $y_2(t)$, and $y_3(t)$.  
\end{enumerate}

\end{problem}

\begin{problem}
 Consider the system of ODEs $y_1' = y_2$ and $y_2'=-9y_1-6y_2$. 
\begin{enumerate}
 \item Write this system in the form $\dfrac{d\vec y}{dt} = A\vec y$. Find the eigenvalues of $A$.  State all the eigenvectors of $A$. 
 \item Why are there not enough eigenvectors to form an invertible matrix $Q$?
 \item The system of ODEs above is equivalent to the ODE $y''+6y'+9y=0$. Explain why.
 \item State a general solution to this ODEs using methods from before.
\end{enumerate}
\end{problem}

We've now seen two examples with repeated eigenvalues.  Sometimes when we see a repeated eigenvalue, we'll be able to get enough linearly independent eigenvectors to form an invertible matrix $Q$. In this case, we'll be able to write $A=QJQ^{-1}$ where $J$ is diagonal. If we can't get enough linearly independent eigenvectors, then we'll have to do something else.  We'll show that we can always write $A=QJQ^{-1}$, where the only nonzero terms off the diagonal are perhaps a few 1's directly above the diagonal.  Luckily, we know how to compute the matrix exponential of this kind of matrix. The matrices $Q$ and $J$ are called a Jordan decomposition for $A$. 

\begin{problem}
 Consider again the system of ODEs $y_1' = y_2$ and $y_2'=-9y_1-6y_2$. 
\begin{enumerate}
 \item We already know the only eigenvalue of the coefficient matrix is $\lambda=-3$. To find the eigenvectors corresponding to $\lambda = -3$, we solve $(A+3 I)\vec x=\vec 0$. Solve this system, and state the solution.
 \item Because the eigenvalue $\lambda = -3$ was repeated, we were hoping to find two linearly independent eigenvalues, but we did not. Instead of solving $(A-\lambda I)\vec x=\vec 0$, let's solve $(A-\lambda I)^2\vec x=\vec 0$.
 
 Find the solutions to $(A-\lambda I)^2\vec x=\vec 0$. Show by hand how to solve this system, and state the solution as linear combination of two independent vectors.
 \item Pick a nonzero vector from the previous part that is not an eigenvector, and call it $\vec v_2$.  Compute $\vec v_1 = (A-\lambda I)\vec v_2$, and show that $\vec v_1$ is an eigenvector.
 \item Let $Q=\begin{bmatrix} \vec v_1 &\vec v_2\end{bmatrix}$. Compute by hand $AQ$ and show that it equals $QJ$ where $J= \begin{bmatrix}-3&1\\0&-3\end{bmatrix}$. 
 \item Now that we have $Q$ and $J$, compute $e^{At}$. The solution to the ODE is $\vec y=e^{At}\vec c$, which you can now compare with your previous problem.
\end{enumerate}
\end{problem}

In three previous problem, we considered the linear system of equations $(A-\lambda I)^2\vec x=\vec 0$. This system produced an extra vector we could use to get an invertible matrix $Q$. The solutions to this system provide us with what we now call generalized eigenvectors.
\begin{definition}[Generalized Eigenvectors]
 The nonzero solutions to the linear system $(A-\lambda I)^1\vec x=\vec 0$ are the eigenvectors of the matrix $A$.  
\begin{enumerate}
 \item If a vector 
$\vec x$ satisfies $(A-\lambda I)^2\vec x=\vec 0$ but does not satisfy $(A-\lambda I)^1\vec x=\vec 0$, then we say that $\vec x$ is generalized eigenvector of order 2. 
 \item If a vector 
$\vec x$ satisfies $(A-\lambda I)^k\vec x=\vec 0$ but does not satisfy $(A-\lambda I)^{k-1}\vec x=\vec 0$, then we say that $\vec x$ is generalized eigenvector of order $k$. 
\end{enumerate}
\end{definition}


\begin{problem}
 Suppose that $A$ has an eigenvalue $\lambda$ and has a generalized eigenvector of order $3$ which we'll call $\vec v_3$. Let $v_2 = (A-\lambda I)\vec v_3$ and $\vec v_1 = (A-\lambda I)\vec v_2$.   
\begin{enumerate}
 \item Explain why $\vec v_1$ is an eigenvector of $A$ and  $\vec v_2$ is a generalized eigenvector of order $2$. 
 \item \marginpar{Remember that $(A-\lambda I)\vec v_3 = \vec v_2$, so we know that $A\vec v_3 = \lambda\vec v_3+\vec v_2$. How will this help you show what is asked for?}
Let $T = \begin{bmatrix} \vec v_1&\vec v_2&\vec v_3\end{bmatrix}$. Show that 
$$A T= 
T
\begin{bmatrix}
\lambda & 1&0\\
0&\lambda & 1\\
0&0&\lambda\\
\end{bmatrix}.
$$
 \item Let $ S = \begin{bmatrix} \vec v_3&\vec v_2&\vec v_1\end{bmatrix}$. Show that 
$$A S = 
S
\begin{bmatrix}
\lambda & 0&0\\
1&\lambda & 0\\
0&1&\lambda\\
\end{bmatrix}.
$$
\end{enumerate}
\end{problem}



\begin{definition}[Jordan Canonical Form]
 Let $A$ be a square matrix. Suppose that $AQ=QJ$, where $Q$ is an invertible matrix and $J$ is a matrix whose only nonzero entries off the diagonal are potentially 1's above the diagonal.  If there is a one in $J$ above the diagonal, then the entries below and to the left of the 1 must be the same. Under these conditions, we call $J$ a Jordan canonical form for $A$.  

 The 1's appear in Jordan canonical form precisely when 
\end{definition}

It is a theorem that every matrix $A$ admits a Jordan canonical form.  The matrix $Q$ consists of eigenvalues and generalized eigenvalues.  Once we have a Jordan canonical form for $A$, the matrix exponential of $At$ is $e^{At} = Q e^{Jt} Q^{-1}$.  

\begin{problem}
Consider the matrix 
$A =
\begin{bmatrix}
 2&1&1\\
 1&2&0\\
 0&0&3
\end{bmatrix}
.$ 
Our goal is to find a Jordan canonical form for $A$ by stating $Q$ and $J$ so that $AQ=QJ$.   
\begin{enumerate}
 \item Show that the eigenvalues of $A$ are $\lambda = 1$ and $\lambda =3$.  The eigenvalue 3 is repeated. Find an eigenvector corresponding to $\lambda =1$, and call it $\vec u$. 
 \item Solve $(A-3I)\vec x=\vec 0$ to find the eigenvectors corresponding to $\lambda = 3$. We're hoping for two independent vectors, but show you only get one.
 \item Now solve $(A-3I)^2\vec x=\vec 0$. Pick a solution that is not an eigenvector and call it $\vec v_2$.  Then compute $\vec v_1 = (A-3 I)\vec v_2.$ 
 \item The vectors $\vec u$, $\vec v_1$, and $\vec v_2$ need to be placed in matrix $Q$. How should you place them?  Make an educated guess, and then compute $Q^{-1}AQ$ to verify if it equals $J$.  If it does not, then try a different order on $Q$. You know you've got the right $J$ when it's almost diagonal with potentially a 1 above and to the right of a repeated eigenvalue. 
\end{enumerate}
\end{problem}

Have you noticed the pattern for finding Jordan form?  Find the eigenvalues. Then find the eigenvectors.  If you don't have enough linearly independent eigenvectors, then you look for generalized eigenvectors of order 2. Continue finding generalized eigenvectors as needed until you get enough linearly independent vectors.  When you select a higher order generalized eigenvector $\vec v$, make sure you compute $(A-\lambda I)^k\vec v$ for $k=1,2,...$ and use those as your lower order vectors.  This will get you Jordan form.  I'll  demo some of this in class.


We've shown that you can start with $A$, and from it determine $Q$ and $J$. We know $AQ = QJ$, which means if we had $Q$ and $J$, we could obtain $A$.  The next problem starts with a $Q$ and $J$. You'll then obtain $A$.  After obtaining $A$, we'll find another $Q$ and $J$ so that $AQ=QJ$.   The $Q$ and $J$ are not unique. 
\begin{problem}
\marginpar{Use software to help you complete this problem. }% 
Let 
$J=
\begin{bmatrix}
 2 & 1 & 0 & 0 \\
 0 & 2 & 0 & 0 \\
 0 & 0 & 3 & 0 \\
 0 & 0 & 0 & 2 \\
\end{bmatrix}
$
and
$Q=
\begin{bmatrix}
 1 & 0 & 2 & -1 \\
 1 & 0 & 0 & 2 \\
 0 & 1 & 1 & -1 \\
 0 & -1 & 0 & 0 \\
\end{bmatrix}
$.
\begin{enumerate}
 \item Compute $A = QJQ^{-1}$ (make sure you use software). Why are the eigenvalues of $A$ equal to 2 and 3? What's the multiplicity of each eigenvalue? For each eigenvalue, give all possible eigenvectors. (Show the appropriate matrices you would rref, state the rref, and then state the eigenvectors.)
 \item The eigenvalue 2 shows up three times, but only contributes two linearly independent eigenvectors.  Find a generalized eigenvector, which we'll call $\vec v_2$, of degree 2 for this eigenvalue (remember to solve $(A-\lambda I)^2\vec x=\vec 0$). Then compute $\vec v_1= (A-2 I)\vec v_2$, which should be an eigenvector. This gives you two vectors corresponding to $\lambda = 2$.  For $\vec v_3$, pick an eigenvector corresponding to $\lambda = 2$ that is not a multiple of $\vec v_1$.  Let $\vec v_4$ be an eigenvector corresponding to the other eigenvalue. 
 \item You now have enough information to state $Q$. It will mostly likely be different than the one you started with.  State $Q$ and then compute $Q^{-1}AQ$ (with a computer) to get $J$.  You should have an almost diagonal matrix with a single 1 above the diagonal. 
 \item State $e^{At}$ (feel free to use a computer to perform any needed matrix multiplications). Check your answer with a computer. 
\end{enumerate}
\end{problem}




\section{Solving Non Homogeneous ODEs}
We now have the tools needed to solve every constant coefficient linear system of ODEs, whether homogeneous or not. They key to solving these problems is a formula we already developed earlier in the semester. If you have forgotten how to find an integrating factor, you may want to review some problems from chapter 4. Then tackle this problem.

\begin{problem}
 Consider the first order ODE $y'-ay=f(t)$. Find an appropriate integrating factor, and then show that a general solution to this ODE is
 $$y(t) = e^{at}c + e^{at}\int_0^t e^{-at}f(t) dt,$$
 where $c$ is an arbitrary constant.  If the ODE is homogeneous with $f(t)=0$, show that $c=y(0)$.
\end{problem}

The solutions above provides a theoretical way to solve every first order linear constant coefficient ODE. If we replace $y$, $c$, and $f$ with vectors, and we replace $a$ with a matrix, then the solution to $\dfrac{d \vec y}{dt} = A\vec y(t)+\vec f(t)$ is simply 
$$\vec y(t) = e^{At}\vec c + e^{At}\int_0^t e^{-At}\vec f(t) dt.$$
\marginpar{It's possible to rework through the details of the problem above to show this is the solution.  I'll leave those details to you. Solving that problem was one of my most exciting discoveries in the last 10 years of teaching.  It's amazing.}
This equation solves just about every ODE we've encountered all semester, and more.  To use this solution, the system must have constant coefficients, but the function $f$ only has to be integrable after multiplying by $e^{-At}$. This greatly extends our ability to solve non homogeneous ODEs. 

\begin{problem} 
Consider the linear system of ODEs given by 
$$y_1' = -3y_1 + y_2 +3\quad \text{and}\quad y_2' = -3y_2+6,$$
with initial conditions $y_1(0)=1$ and $y_2(0)=0$. 
\begin{enumerate}
 \item Write this linear system in the form $\frac{d\vec y}{dt} - A\vec y = \vec f(t)$. [Hint: The right hand side might be constant.]
 \item Explain how to compute $e^{At}$ and $e^{-At}$.  What are $[e^{At}]^{-1}$ and $e^{A(-t)}$?
 \item Compute, by hand, $\int_0^t e^{-At}\vec f(t) dt$, and then with a computer give the product $e^{At}\int_0^t e^{-At}\vec f(t) dt$. 
 \item We know the general solution is $\vec y(t) = e^{At}\vec c + e^{At}\int_0^t e^{-At}\vec f(t) dt.$  Use the initial conditions to find $\vec c$. Show us what matrix you are row reducing. 
 \item You should have ${y_2}(t)=2-2 e^{-3 t}$.  What is ${y_1}(t)$?
\end{enumerate}
\end{problem}

\begin{problem} 
Consider the linear system of ODEs
$$
\frac{d\vec y}{dt}
=
\begin{bmatrix}
 2&3\\1&4
\end{bmatrix}
\vec y +
\begin{bmatrix}
 0\\5t
\end{bmatrix}
,$$
with initial conditions
$y_1(0)=4$, $y_2(0)=0$.
\begin{enumerate}
 \item Use a computer to give the eigenvalues and an eigenvector for each eigenvalue.  State $Q$ and $J$ so that $AQ=QJ$ where $J$ is a Jordan form for $A$. 
 \item State $e^{Jt}$. What matrix product would you compute to get $e^{At}$? State the matrices you would multiply, and then state both $e^{At}$ and $e^{-At}$. Please use a computer to do all this.
 \item Compute, by hand, $\int_0^t e^{-At}\vec f(t) dt$, and then with a computer give the product $e^{At}\int_0^t e^{-At}\vec f(t) dt$. 
 \item Show how to use the initial conditions to find $\vec c$ in $\vec y(t) = e^{At}\vec c + e^{At}\int_0^t e^{-At}\vec f(t) dt.$
\end{enumerate}
\end{problem}

\begin{problem}
Consider the linear system of ODEs
$$
\frac{d\vec y}{dt}
=
\begin{bmatrix}
 0&1\\-1&-2
\end{bmatrix}
\vec y +
\begin{bmatrix}
 e^{-t}\\0
\end{bmatrix}
.$$
\begin{enumerate}
 \item Find $Q$ and $J$ so that $J$ is a Jordan form for $A$.  You'll need to get a generalized eigenvector.  Show how you did this. 
 \item State $e^{At}$ and $e^{-At}$.
 \item Compute, by hand, $\int_0^t e^{-At}\vec f(t) dt$, and then with a computer give the product $e^{At}\int_0^t e^{-At}\vec f(t) dt$. 
 \item If you let $\vec c=\vec 0$, then what are $y_1$ and $y_2$.  This is what we called $y_p$ in the non homogeneous ODE section. 
\end{enumerate}
\end{problem}

You've now seen the key ideas needed to solve any kind of system. The computations can get quite intense, but we can program a computer to take care of all the computations.  

%Review concept of superposition.  Define Wronskian and state corresponding theorem.  Have them compute the Wronskian for several problems (showing things are linearly independent or dependent). (We'll skip the Wronskian this time.)

%Finish the chapter by setting up ODE problems for mixing tanks, mass spring systems, and electrical networks. Use a computer to solve them all, stating $Q$, $J$, $e^{Jt}$, $e^{At}$, and $\vec c$ (from IC). This might be 4 more problems.

\section{Applications}
Let's finish this chapter with some application problems.  You don't have to be able to compute the matrix exponential problems to solve these problems, so if you got stuck above, please do these. 

\begin{problem} 
Suppose we have three large tanks containing various amounts of salt. \marginpar{We don't have to use salt.  This could represent 3 different countries and products they wish to import/export.  It could be sewage at a waste transfer station. We might consider three countries and the spreading of a virus. We could look at three cities and the flow of traffic. The applications are endless.}
The tanks have volumes $V_1=30$ gal, $V_2=20$ gal, and $V_3=50$ gal.  
\begin{itemize}
 \item An inlet valve pumps $5$ gallons of water into tank 1 each minute. The water coming in contains 3 lbs of salt per gallon. This water is being added to the system from some external source.   
 \item Tank 1 has an outlet value that pumps 9 gallons per minute out to tank 2. Tank 1 has an inlet value that receives 3 gallons per minute from tank 2, and 1 gallon per minute from tank 3. In all, this mean that tank 1 has 9 gallons coming in per minute, and 9 gallons going out per minute.
 \item Tank 2 receives 9 gallons per minute from tank 1.  Of those 9 gallons, it sends 3 gallons per minute to tank 1 and 4 gallons per minute to tank 2.  The other 2 gallons per minute leak out the top (through a crack).
 \item Tank 3 receives 4 gallons per minute from tank 2.  It sends 1 gallon per minute back to tank 1, and then the remaining 3 gallons per minute are sent out a hose to some external spot.  
\end{itemize}
Assume the initial salt content is zero in each tank.  As time moves on, the salt that is added to tank 1 will eventually reach the other tanks.  After some time has elapsed, how much salt will be in each tank? %We have all the tools needed to answer this question. 
\begin{enumerate}
 \item Let $y_1$, $y_2$, and $y_3$ be the lbs of salt in each tank after $t$ minutes.  Write this tank mixing problem as a linear system of ODEs in the form $\frac{d\vec y}{dt} - A\vec y = \vec f(t)$.
 \item Use software to completely solve the system. State $y_1(t)$, $y_2(t)$, and $y_3(t)$. 
 \item Let $t\to \infty$. What will be the salt content in each tank in the long run.
\end{enumerate}
\end{problem}

\begin{problem}
Consider the following mechanical system.  Attach a spring to the top of the ceiling. Add an object with mass $m_1$ to the bottom of the spring. We'll assume the spring's mass is negligible. The spring constant is $k_1$, and the coefficient of friction is $c_1$. To the bottom of the first mass, we attach a second spring, and hang another object to the end of the second spring.  The second object has mass $m_2$. The second spring has negligible mass, with spring constant $k_2$ and coefficient of friction $c_2$. Assume that this mechanical system is currently stabilized, so neither mass is moving.  We'll let $y_1(t)$ be the position of the first mass, relative to this equilibrium position (so if $y_1(t)=3$, then we'd be 3 cm above the equilibrium point).  We'll let $y_2(t)$ be the position of the second mass relative to its equilibrium position.  We displace the objects from equilibrium, and let them go, so we have the initial conditions $y_1(0)=a$, $y_2(0)=b$, $y'_1(0)=0$, and $y'_2(0)=0$. Our goal is to predict the future, namely give the position of both springs at time $t$.   
\begin{enumerate}
 \item Start by assuming that both $c_1=0$ and $c_2=0$. 
Explain why $m_1y_1'' = -k_1 y_1 +k_2(y_2-y_1)$. Obtain a similar equation for $my_2''$.
 \item Let $v_1=y_1'$ and $v_2=y_2'$. This allows us to replace $y_1''$ with $v_1'$, and $y_2''$ with $v_2'$, and then we have a system of first order linear ODEs.  Write this system in the matrix form 
$$
\begin{pmatrix}
 y_1\\y_2\\v_1\\v_2
\end{pmatrix}'
=
\begin{bmatrix}
 ?&?&?&?\\
 ?&?&?&?\\
 ?&?&?&?\\
 ?&?&?&?
\end{bmatrix}
\begin{pmatrix}
 y_1\\y_2\\v_1\\v_2
\end{pmatrix}.
$$
\item Now assume that $c_1$ and $c_2$ are not zero.  How does this change the ODEs from part 1?  What's the corresponding 4 by 4 matrix in part 2?
\end{enumerate}
%Use Larry's springs.  Ask him for the coefficients.  Have the displacement go up (not down).
\end{problem}












Our last application involves modeling the current in an electrical system with two loops. A similar computation will work with any number of loops, though the number of loops causes the size of the system to increase quite rapidly.
Remember that Kirchoff's current law states that at each node, the current in equals the current out.  In addition, Kirchoff's voltage laws states that along each loop, the voltage supplied equals the voltage suppressed. Each resistor contributes a voltage drop of $RI$ ohms, each capacitor a drop of $\frac{1}{C}\int I dt$ farads, and each inductor a voltage drop of $LI^\prime$ Henrys.

\begin{problem}
\marginpar{\input{electric-circuit-2-loops-RCL}}%
Consider the electrical network on the right.  
Kirchoff's current law states that $I_1=I_2+I_3$.  On the left loop, Kirchoff's voltage law states that $E = R_1I_1+R_2I2+LI_1^\prime$.  On the right loop, Kirchoff's voltage law states that $0=I_3R_3 +\frac{1}{C}\int I_3 dt - R_2I_2$.   Solve this system for $I_1^\prime, I_2^\prime,$ and $I_3^\prime$, and then write the system in matrix form $\ds \frac{d\vec I}{dt} = A\vec I +\vec f(t)$, i.e. in the form 
$$
\begin{pmatrix}
 I_1\\I_2\\I_3
\end{pmatrix}'
=
\begin{bmatrix}
 ?&?&?\\
 ?&?&?\\
 ?&?&?
\end{bmatrix}
\begin{pmatrix}
 I_1\\I_2\\I_3
\end{pmatrix}
+
\begin{pmatrix}
 ?\\?\\?
\end{pmatrix}
.$$
You'll need to differentiate the first and last equation to get an $I_1'$ and $I_3'$.  Once you've got the system set up, a computer will give the currents almost instantly. This algorithm is coded into any electrical network software package.
\end{problem}
























%In all the problems above,  we've solved a linear system of ODEs in the form $\vec y' = A\vec y$. We've stuck to only 2 by 2 matrices, but the ideas here work on larger system. The matrix $Q$ above depends entirely on the eigenvalues of the matrix. The columns of $Q$ are the eigenvectors of $A$. If we can find enough linearly independent eigenvectors, then the matrix $J$ is a diagonal matrix, and the entries of $J$ are precisely $e^If you can find two linearly independent eigenvectors for the coefficient matrix $A$, 




\note{

This section needs major revision.

The students are weak with matrix multiplication symbolically.  The concept of noncommutative doesn't make sense. They are unable to solve for A or J in AQ=QJ correctly.  They have never had to worry about whether you multiply on the left or the right.  That needs to change (but has to change earlier in the semester).   

They are also very uncomfortable symbolic manipulations. $A^k$ made no sense.  They were unable to do this.  The generalized eigenvector part was pretty rough.  I never introduced the words ``Diagonalize''.  That might be best done in chapter 3.  

I could go way back to chapter 3 and add in inner products before I show them linear regression.  Maybe. Grahm schmidt would be nice. 

I guess the real questions is, ``How much time should I spend on linear algebra.'' If spend enough time on linear algebra, then the entire ODE stuff is done automatically with the matrix exponential.  It wouldn't use Laplace as much at all, but it would give solutions rather quickly.  

If I'm happy letting them use software, then I could give them the matrix exponential approach really early on, and that's how we would solve stuff.  I could introduce the other techniques along the way.  How much do they really need the method of undetermined coefficients?  Is it needed?  I could get all this straight from the matrix exponential.

I have to rethink the entire course.

What should I do?  What should I teach? I want them to use Laplace transforms early on. Maybe I should just start with ODEs for while, then add in matrices, and emphasize matrices the rest of the semester.  Who knows.  This could be fun.


Bonnie? Larry?  Dave B? Dave Stowell?

You can get to Laplace transforms by inquiry if you just solve the general first order linear ODE.  The transform sits in the problem.  Maybe that's the way to go.  Who knows.  This could be a fun adventure.


I still need to add in the Wronskian.  I missed it this semester.

\section{Definitions and Theory for Systems of ODEs}
We will focus on linear systems. We say a linear system $\vec y^\prime = A\vec y +\vec g$ is homogeneous if $\vec g =\vec 0$ ($\vec y^\prime -A\vec y =\vec 0$). The superposition principle says that any linear combination of two solutions to a homogeneous linear system of ODEs is again a solution.  A basis of solutions, written $\vec y_{1},\vec y_{2},\ldots, \vec y_{n}$, is a set of $n$ linearly independent solutions. The general solution is $\vec y = c_1\vec y_{1}+c_2\vec y_{2}+\cdots+c_n \vec y_{n} = Y\vec c$, where $Y$ is a matrix whose columns are the vectors $\vec y_{i}$.  The Wronskian of $n$ solutions is the determinant of $Y$ (the matrix whose columns are the solutions). If the Wronskian is ever zero on an interval, then it is identically zero and the solutions are dependent.  If it is nonzero at a point in an interval, then the solutions are linearly independent.

}