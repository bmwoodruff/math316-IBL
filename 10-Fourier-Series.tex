

This chapter covers the following ideas. When you create your lesson plan, it should contain examples which illustrate these key ideas. Before you take the quiz on this unit, meet with another student out of class and teach each other from the examples on your lesson plan. 

\begin{enumerate}
\item Define period, and how to find a Fourier series of a function of period $2\pi$ and $2L$. 
\item Explain how to find Fourier coefficients using Euler formulas, and be able to explain why the Euler formulas are correct.
\item Give conditions as to when a Fourier series will exist, and explain the difference between a Fourier series and a function at points of discontinuity.
\item Give examples of even and odd functions, and correspondingly develop Fourier cosine and sine series.  Use these ideas to discuss even and odd half-range expansions.
\end{enumerate}



\section{Fourier Series}

In the power series unit, we showed how to write function $f(x)$ as a power series $\ds\sum_{n=0}^{\infty}a_n x^n$. The radius of convergence tells us precisely for which $x$ values the power series will converge to the function $f(x)$. We call it a power series because we use powers of $x$ as the functions we are summing. At any finite stage, we are trying to approximate the function $f(x)$ using linear combinations of power of $x$. We can find the coefficients $a_n$ with the  formula $a_n=\dfrac{f^(n)(0)}{n!}$.  

Are there other kinds of series? Do we have to use powers of $x$?  
In class, we discussed how to use Legendre polynomials to do the same thing.  
If we let $P_n(x)$ be the $n$th degree Legendre polynomial, then we could write $\ds f(x)=\sum_{n=0}^\infty a_n P_n(x)$. 
We showed that on the interval $[-1,1]$, the integral $\int_{-1}^1 P_n(x)P_m(x)dx=0$ if $n\neq m$.  We then used this to show that we can compute the coefficients $a_n$ using the formula 
$$\ds a_n = \frac{\int_{-1}^1 f(x)P_n(x)dx}{\int_{-1}^1 P_n(x)P_n(x)dx}.$$ 
The key to using a Legendre series is the fact that $$\int_{-1}^1 P_n(x)P_m(x)dx=0.$$ 
This integral equation is a generalization of the dot product.  In the dot product, we multiply corresponding components together and then sum.  That's precisely what the integral above does.  Let's make a definition.

\begin{definition}[Inner Product]
Let $f(x)$ and $g(x)$ be bounded functions over some interval $[a,b]$. The inner product of $f$ and $g$ over $[a,b]$ is the integral  
$$\left<f,g\right> = \int_a^b f(x)g(x) dx.$$ 
This inner product is a generalization of the dot product.  We say that two functions are orthogonal over $[a,b]$ if their inner product over $[a,b]$ is equal to zero.
\end{definition}

\begin{problem}
Consider the function $f(x) = x$ and $g(x)=x^2$. 
\begin{enumerate}
 \item Compute the inner products $\left<f,f\right>$, $\left<g,g\right>$, and $\left<f,g\right>$. [For the first, you just need to compute the integral $\int_0^1 x\cdot x dx$. ]
 \item Recall that $\vec u\cdot \vec u = |u|^2$. So the length of a vector is the square root of the dot product of the vector with itself. We'll define the length of a function to be the square root of the inner product of a function with itself, and write $||f|| = \sqrt{\left<f,f\right>}.$ What is the length of $f$ and the length of $g$ over the interval $[0,1]$. [Hint, you already did the inner products in part 1.  
 \item Recall that $\vec u\cdot \vec v = |\vec u||\vec v|\cos\theta$, where $\theta$ is the angle between the vectors $\vec u$ and $\vec v$. Use this idea to invent a definition for the angle between two functions, and then use your definition to compute the angle between $f$ and $g$.  
\end{enumerate}
\end{problem}


With the word inner product, we can now talk about the ``dot product'' of functions.  We can use the words Length and angle when talking about functions.  We obtained the Legendre series coefficients because the inner product of any two different Legendre polynomials, over the interval $[-1,1]$, is zero. The Legendre polynomials form an infinite collection of orthogonal vectors. We've got a 90 degree angle between any two functions. We can use these polynomials to approximate any other function by considering linear combinations of these orthogonal functions.  This is one of the big ideas in modern science.  Rather than working with complicated functions $f$, we can approximate them with linear combinations of simpler functions.  What constitutes a ``simple'' function depends on the problem.  Legendre polynomials show up when studying spherically symmetric problems.  Bessel functions shows up when studying radially symmetric problems. Sometimes the hardest part about solving a problem is trying to determine which ``simple'' collection of functions to use.    

Fourier was one of the first to use something other than power series to approximate a function. He did it while trying to understand how heat flowed in a cannon.  Napoleon asked Fourier to study this problem, because cannons were exploding on his soldiers because the cannon balls would not leave a cannon after heat had caused the barrel to expand. Fourier discovered that you can use sine and cosine series to approximate a function.  That's what we'll study in this chapter.  In the work below, the variable $L$ will stand for the length of a cannon.  

\begin{problem}
 Let $f_n(x) = \cos\left(\dfrac{n\pi x}{L}\right)$ for $n=0,1,2, \ldots$.  So we have $f_0(x) = \cos\left(\dfrac{0\pi x}{L}\right)$, $f_1(x) = \cos\left(\dfrac{\pi x}{L}\right)$, $f_2(x) = \cos\left(\dfrac{2\pi x}{L}\right)$, $f_3(x) = \cos\left(\dfrac{3\pi x}{L}\right)$, etc.  This is an infinite collection of functions. Our goal is to show that on the interval $[0,L]$, the functions $f_n$ and $f_m$ are orthogonal, provided $n\neq m$. To do this, you'll need to prove that any pair has an inner product of 0. 
\begin{enumerate}
 \item Draw the functions $f_0(x) = \cos\left(\dfrac{0\pi x}{L}\right)$, $f_1(x) = \cos\left(\dfrac{\pi x}{L}\right)$, and $f_2(x) = \cos\left(\dfrac{2\pi x}{L}\right)$ over the interval $[0,L]$.
 \item Compute the integral $\int_0^L f_0(x) f_n(x)dx$ for $n\neq 0$.  What is $\left<f_0,f_n\right>$?
 \item Now show $\int_0^L f_n(x) f_m(x)dx=0$ for $n\neq m$, with $n,m>0$. As a hint, you'll want to look up a product-to-sum trig identity, after which this integral is quickly doable. 
\end{enumerate}
\end{problem}

 We now know that the set of functions $f_n(x) = \cos\left(\dfrac{n\pi x}{L}\right)$ forms an orthogonal set of functions over the interval $[0,L]$. Let's now make a Fourier cosine series for a function.

\begin{problem}
 Suppose that $f(x)$ is defined on $[0,L]$. If we assume that $$f(x) = \sum_{n=0}^\infty A_n\cos\left(\dfrac{n\pi x}{L}\right),$$ 
 then show that 
$$A_n = \ds\frac{\int_0^Lf(x)\cos\left(\dfrac{n\pi x}{L}\right)dx}{\int_0^L\cos^2\left(\dfrac{n\pi x}{L}\right)dx}.$$ 
 Then compute the bottom integral to show that for $n\geq 1$ we have 
$$A_n = \ds\frac{2}{L}\int_0^Lf(x)\cos\left(\dfrac{n\pi x}{L}\right)dx,$$
and if $n=0$ then we have 
$$A_0 = \ds\frac{1}{L}\int_0^Lf(x)dx.$$
 [Hint:  Multiply both sides by $\cos\left(\dfrac{m\pi x}{L}\right)$ and then integrate from $0$ to $L$.  You can use the orthogonality of cosines to solve for the coefficients. The integral of the bottom will involve a cosine half angle formula.]
\end{problem}

We could have repeated the previous two problems with the sine function, and would have gotten very similar results.  Based of the previous two problems, let's make a formal definition.

\begin{definition}[Fourier Sine and Cosine Series]
 Let $f(x)$ be a function defined on $[0,L]$. We define the Fourier sine series and Fourier cosine series of $f$ (on $[-L,L]$) to be the series
$$\sum_{n=1}^\infty B_n \sin\left(\frac{n\pi x}{L}\right) \quad\text{ and }\quad \sum_{n=0}^\infty A_n \cos\left(\frac{n\pi x}{L}\right),$$
respectively.  The coefficients above are given by the formulas
$$B_n=\frac{2}{L}\int_{0}^L f(x)\sin\left(\frac{n\pi x}{L}\right)dx
\quad\text{ and }\quad 
\begin{array}{l}
\ds A_0=\frac{1}{L}\int_{0}^L f(x)dx,\\
\ds A_n=\frac{2}{L}\int_{0}^L f(x)\cos\left(\frac{n\pi x}{L}\right)dx.
\end{array}$$
\end{definition}


\begin{problem}
 Let $f(x) = x$ over the interval $[0,L]$. Draw the function $f$.  Then compute the Fourier sine series of $f(x)$ by computing the integrals 
$$B_n=\frac{2}{L}\int_{0}^L f(x)\sin\left(\frac{n\pi x}{L}\right)dx.$$
 Write out the first 4 nonzero terms of the series by writing
$$ \sum_{n=1}^\infty B_n \sin\left(\frac{n\pi x}{L}\right) = 
B_1 \sin\left(\frac{\pi x}{L}\right)+
B_2 \sin\left(\frac{2\pi x}{L}\right)+
B_3 \sin\left(\frac{3\pi x}{L}\right)+
\cdots.$$
Be prepared to explain how you use integration by parts to compute the integrals. 
\end{problem}

\begin{problem}
 Let $f(x) = x$ over the interval $[0,L]$. Draw the function $f$.  Then compute the Fourier cosine series of $f(x)$ by computing the integrals 
$$A_0=\frac{1}{L}\int_{0}^L f(x)dx \quad\text{and}\quad A_n=\frac{2}{L}\int_{0}^L f(x)\cos\left(\frac{n\pi x}{L}\right)dx.$$
 Write out the first 4 nonzero terms of the series by writing
$$ \sum_{n=0}^\infty A_n \cos\left(\frac{n\pi x}{L}\right) = 
A_0 +
A_1 \cos\left(\frac{\pi x}{L}\right)+
A_2 \cos\left(\frac{2\pi x}{L}\right)+
A_3 \cos\left(\frac{3\pi x}{L}\right)+
\cdots.$$
Be prepared to explain how you use integration by parts to compute the integrals. 
\end{problem}

\begin{problem}
 In the previous two problems, you computed the Fourier cosine and Fourier sine series of $f(x)=x$ over the interval $[0,L]$. This problem will repeat the above with technology, and then you'll graph your results. To allow software to create graphs, you'll have to pick a length $L$ that's an actual number (maybe try $L=7$).  
\begin{enumerate}
 \item Use software to obtain the first 4 nonzero terms of the Fourier sine and Fourier cosine series of $f(x)=x$ over $[0,L]$.
 \item Use software to graph both $f(x)=x$ and the first four nonzero terms of the Fourier sine series. 
Have the bounds of your graph be  over the interval $[-3L, 3L]$. 
 \item Use software to graph both $f(x)=x$ and the first four nonzero terms of the Fourier cosine series.
Have the bounds of your graph be  over the interval $[-3L, 3L]$. 
 \item Make some conjectures about what you see in your graphs.
\end{enumerate}
To present this in class, you should come to class with your graphs already printed out.
\end{problem}














\begin{definition}[Piecewise Smooth]
 We say that  function $f(x)$ is smooth on an interval $[a,b]$ if the function and its derivative are both bounded and continuous on $(a,b)$.  We say that a function $f(x)$ is piecewise smooth on an interval $(a,b)$ if the interval can be partitioned into a finite number of pieces and on each piece the function $f(x)$ is smooth (so $f'(x)$ may not exist at finitely many points).  
\end{definition}


\begin{definition}[Fourier Series]
 Let $f(x)$ be a function defined on $[-L,L]$ such that the Fourier coefficients
\begin{align*}
a_0&=\frac{1}{2L}\int_{-L}^L f(x)dx,\\
a_n&=\frac{1}{L}\int_{-L}^L f(x)\cos\left(\frac{n\pi x}{L}\right)dx \text{, and}\\
b_n&=\frac{1}{L}\int_{-L}^L f(x)\sin\left(\frac{n\pi x}{L}\right)dx 
\end{align*}
exist. We define the Fourier series of $f$ over the interval $[-L,L]$ to be the formal infinite series
$$a_0+\sum_{n=1}^\infty a_n \cos\left(\frac{n\pi x}{L}\right) +\sum_{n=1}^\infty b_n \sin\left(\frac{n\pi x}{L}\right).$$
Regardless of whether or not the series converges, we will write
$$f(x)\sim a_0+\sum_{n=1}^\infty a_n \cos\left(\frac{n\pi x}{L}\right) +\sum_{n=1}^\infty b_n \sin\left(\frac{n\pi x}{L}\right).$$
\end{definition}

\begin{remark}
 In the definition above, we do not require the Fourier series to actually converge to $f$. The following theorem, which we will give without proof, provides the needed conditions for the series to converge.  We will use this proof (which requires some real analysis) to prove other facts about Fourier series throughout this chapter.
\end{remark}

\begin{theorem}[Fourier's Theorem]
 Suppose $f(x)$ is piecewise smooth on the interval $-L\leq x\leq L$. Then the Fourier series of $f(x)$ converges to the period extension of $f$ at the points where $f$ is continuous.  If the periodic extension of $f$ is not continuous at a point $x$, then the Fourier series converges to the average of the left and right limits at $x$, namely $\ds\frac{f(x+)+(f(x-)}{2}$.
\end{theorem}




\begin{problem}
 Let $f(x) = x$ over the interval $[-L,L]$. Compute the Fourier series of $f(x)$ by computing the integrals for $a_0$, $a_n$, and $b_n$.  
Show your integration steps.
\end{problem}

\begin{problem}
 Let $f(x) = x^2$ over the interval $[-L,L]$. Compute the Fourier series of $f(x)$ by computing the integrals for $a_0$, $a_n$, and $b_n$
Show your integration steps.
\end{problem}

\begin{problem}
 Let $f(x) = e^x$, $g(x)=\cosh(x)$ and $h(x)=\sinh(x)$ over the interval $[-L,L]$. Use software to compute the Fourier coefficients for each function.  What patterns do you see?  You are welcome to just state the Fourier coefficients for each (you don't have to show your integration steps). 
\end{problem}









Notice that there are other kinds of series.  In class, I showed you how to obtain a function as an infinite sum of Legendre polynomials. We also introduced the concept of orthogonality of functions, and showed that the Legendre polynomials formed an infinite collection of orthogonal functions. We also showed how to write a function $f$ as a series of Legendre polynomials.  

We can do the exact same thing with trig functions.  We'll develop the cosine series, the sine series, and then the Fourier series. We'll compute these series for a few functions.  After that let's look at how these were developed by Fourier, as he studied the heat equation.

Maybe we should start with why we care about something that has length L. Then get to the reason why we look at $n\pi/L$.  Let's definitely do that.   

Then prove orthogonality of the functions on these intervals.

Show that if we write a function as a series, and the functions are orthogonal, then we know the coefficients through an integral formula.

Obtain a formula for the cosine series coefficients and the sine series coefficients.

Find the Fourier coefficients of a cosine series. Draw the function and several terms of the series. 

Find the Fourier coefficients of a sine series. Draw the function, and several terms of the series. 

What are the coefficients of a Fourier series. Find the coefficients for a function.

If a function is odd or even, what can be said.

I can grab a lot of these problems from my PDE course notes. 





\note{

\section{Basic Definitions}
A function $f(x)$ is said to be periodic with period $p$ if $f(x+p)=f(x)$ for all $x$ in the domain of $f$. This means that the function will repeat itself every $p$ units. The trig functions $\sin x$ and $\cos x$ are periodic with period $2\pi$, as well as with period $4\pi, 6\pi,8\pi$, etc. The fundamental period is the smallest positive period of a function. The function $\sin nx$ is periodic, with fundamental period $\frac{2\pi}{n}$, though it also has period $2\pi$.  

If two functions are period with the same period, then any linear combination of those functions is periodic with the same period. In particular, the sum $a_0 + \sum_{n=1}^\infty (a_n\cos nx +b_n\sin nx)$ has period $2\pi$.  This sum is called a Fourier series, where $a_i,b_i$ are called Fourier coefficients.  Given a function $f(x)$ which has period $2\pi$, we write $$f(x) = a_0 + \sum_{n=1}^\infty (a_n\cos nx +b_n\sin nx)$$ where the Fourier coefficients of $f(x)$ are given by the Euler formulas $$a_0 = \frac{1}{2\pi}\int_{-\pi}^\pi f(x)dx, a_n =  \frac{1}{\pi}\int_{-\pi}^\pi f(x)\cos (nx) dx,b_n =  \frac{1}{\pi}\int_{-\pi}^\pi f(x)\sin (nx) dx$$ for $n\geq 1$. This is another way of expressing a function in terms of an infinite series. 
A Fourier series will converge to the function $f(x)$ for a function which is piecewise continuous and has a left and right hand derivative at each point of the domain. At a point of discontinuity, the Fourier series will converge to the average of the left and right limits at that point. One main use of Fourier series is in solving partial differential equations.

If the function has period $2L$ instead of period $2\pi$, then we make a substitution in the formulas above. Replace every $x$ above with $X$, and then perform the substitution $\frac{X}{2\pi}=\frac{x}{2L}$, or $X=\frac{\pi}{L}x$. The function $\sin\frac{\pi x}{L}$ has period $2L$, and the Fourier series becomes $$a_0 + \sum_{n=1}^\infty \left(a_n\cos \frac{n\pi x}{L} +b_n\sin \frac{n\pi x}{L}\right)$$ where the Fourier coefficients are given by $$a_0 = \frac{1}{2L}\int_{-L}^L f(x)dx, a_n =  \frac{1}{L}\int_{-L}^L f(x)\cos  \frac{n\pi x}{L} dx,b_n =  \frac{1}{L}\int_{-L}^L f(x)\sin  \frac{n\pi x}{L} dx.$$

\subsection{Examples}
Let $f(x) = \begin{cases}1&0<x<\pi\\-1&-\pi<x<0\end{cases}$. The function $f(x)$ has period $2\pi$.  We compute $a_0 
= \frac{1}{2\pi}\int_{-\pi}^\pi f(x)dx 
= \frac{1}{2\pi}\left(\int_{-\pi}^0 -1dx+\int_{0}^\pi 1dx\right) 
= \frac{1}{2\pi}\left(-\pi+\pi\right) 
= 0$.  
Also, we have 
\begin{center}
\begin{tabular}{ll}
$\begin{array}{rl}
a_n 
&=  \frac{1}{\pi}\int_{-\pi}^\pi f(x)\cos (nx) dx\\
&=  \frac{1}{\pi}\left(\int_{-\pi}^0 - \cos (nx) dx + \int_{0}^\pi \cos (nx) dx\right)\\
&=  \frac{1}{\pi}\left( -\frac{\sin nx}{n} \big|_{-\pi}^0 + \frac{\sin nx}{n} \big|_{0}^\pi\right)\\
&=  \frac{1}{\pi}( 0-0 + 0-0)\\
&=0
\end{array}$
&
$\begin{array}{rl}
b_n 
&=  \frac{1}{\pi}\int_{-\pi}^\pi f(x)\sin (nx) dx\\
&=  \frac{1}{\pi}\left(\int_{-\pi}^0 - \sin (nx) dx + \int_{0}^\pi \sin (nx) dx\right)\\
&=  \frac{1}{\pi}\left( \frac{\cos nx}{n} \big|_{-\pi}^0 - \frac{\cos nx}{n} \big|_{0}^\pi\right)\\
&=  \frac{1}{\pi}( \frac{1}{n}-\frac{\cos n\pi}{n} - \frac{\cos n\pi}{n}+1)\\
&=  \frac{1}{n\pi}(2-2\cos n\pi)
\end{array}$
\end{tabular}
\end{center}
If $n$ is even then $\cos n\pi = 1$. If $n$ is odd then $\cos n\pi = -1$. Hence $b_n = \frac{4}{n\pi}$ if $n$ is odd and $b_n=0$ if $n$ is even. This means that we can write $$f(x)=\begin{cases}1&0<x<\pi\\-1&-\pi<x<0\end{cases}=0+\sum_{n=0}^\infty \frac{4}{n\pi}\sin n\pi = \frac{4}{\pi}\left(\sin x + \frac{1}{3}\sin 3x + \frac{1}{5}\sin 5 x + \frac{1}{7}\sin 7x +\cdots\right).$$

The Fourier series of $f(x)=af_1(x)+bf_2(x)$ is found by computing the Fourier series of $f_1$ and $f_2$, multiplying by $a$ and $b$ and then adding.  The function $f(x) =\begin{cases}2&0<x<\pi\\0&-\pi<x<0\end{cases}$ is the same as $f_1(x) +f_2(x)$, where $f_1(x) = \begin{cases}1&0<x<\pi\\-1&-\pi<x<0\end{cases}$ and $f_2(x) = 1$.  The Fourier series of $f_1(x)$ was computed above, and the Fourier series $f_2$ has coefficients $a_0=1,a_n=b_n=0$, so its Fourier series is simply $f_2(x) = 1 $.  Hence this gives the Fourier series $$\begin{cases}2&0<x<\pi\\0&-\pi<x<0\end{cases}=1+\sum_{n=0}^\infty \frac{4}{n\pi}\sin n\pi = 1+\frac{4}{\pi}\left(\sin x + \frac{1}{3}\sin 3x + \frac{1}{5}\sin 5 x + \frac{1}{7}\sin 7x +\cdots\right).$$ 
Division by 2 gives the Fourier series  $$\begin{cases}1&0<x<\pi\\0&-\pi<x<0\end{cases}=\frac12+\sum_{n=0}^\infty \frac{2}{n\pi}\sin n\pi = \frac12+\frac{2}{\pi}\left(\sin x + \frac{1}{3}\sin 3x + \frac{1}{5}\sin 5 x + \frac{1}{7}\sin 7x +\cdots\right).$$


We now consider a function with period 8  defined by the $f(x) = \begin{cases}0&-4<x<-2\\1&-2<x<2\\0&2<x<4\end{cases}$. This function is 1 for $-2<x<2$, $6<x<10$, etc.  It is a regular pulse which is on for 4 units of time, and then off for four units of time.  Since the period is not $2\pi$, but instead $2L=8$, we have $L=4$.  The Fourier coefficients are $a_0 = \frac{1}{2(4)}\int_{-4}^4f(x)dx = \frac{1}{8}\int_{-2}^2 1dx = \frac{1}{2}$, and 
\begin{center}
\begin{tabular}{ll}
$\begin{array}{rl}
a_n 
&=  \frac{1}{L}\int_{-4}^4 f(x)\cos \frac{n\pi x}{4} dx\\
&=  \frac{1}{4}\int_{-2}^2 \cos \frac{n\pi x}{4} dx \\
&=  -\frac{1}{4}\frac{4}{n\pi}\sin \frac{n\pi x}{4} \big|_{-2}^2 \\
&=  -\frac{1}{n\pi}\sin \frac{n\pi x}{4} \big|_{-2}^2 \\
&=  -\frac{1}{n\pi}\left(\sin \frac{ 2n\pi }{4}-\sin \frac{-2 n\pi }{4}\right)\\
&=  \frac{1}{n\pi}(2\sin \frac{ n\pi }{2}) 
\end{array}$
&
$\begin{array}{rl}
b_n 
&=  \frac{1}{L}\int_{-4}^4 f(x)\sin \frac{n\pi x}{4} dx\\
&=  \frac{1}{4}\int_{-2}^2 \sin \frac{n\pi x}{4} dx \\
&=  -\frac{1}{4}\frac{4}{n\pi}\cos \frac{n\pi x}{4} \big|_{-2}^2 \\
&=  -\frac{1}{n\pi}\cos \frac{n\pi x}{4} \big|_{-2}^2 \\
&=  -\frac{1}{n\pi}\left(\cos \frac{ n\pi }{2}-\cos \frac{- n\pi }{2}\right)\\
&=  \frac{1}{n\pi}(0) =0
\end{array}$
\end{tabular}
\end{center}
If $n$ is even, then $a_n=0$ as sine is 0 at integer values. We have $a_1=\frac{2}{n\pi} = a_5=a_9=\cdots$, and $a_3=-\frac{2}{n\pi} = a_7=a_{11}=\cdots$. Hence the Fourier series is
$$f(x) = \begin{cases}0&-4<x<-2\\1&-2<x<2\\0&2<x<4\end{cases}= \frac12+\frac{2}{\pi}\left(\cos \frac{\pi x}{4} - \frac{1}{3}\cos  \frac{3 \pi x}{4} + \frac{1}{5}\cos  \frac{5 \pi x}{4} - \frac{1}{7}\cos  \frac{7 \pi x}{4} +\cdots\right).$$


\section{Orthogonality of Trigonometric functions}
For any integers $m\neq n$, we have $\int_{-\pi}^\pi \cos nx \cos mx dx =0, \int_{-\pi}^\pi \sin nx \sin mx dx =0, \int_{-\pi}^\pi \sin nx \cos mx dx =0$.  In addition, if $m=n$ then $\int_{-\pi}^\pi \sin nx \cos nx dx =0$.  Because these integrals are zero, we say that $\sin nx, \cos mx$ forms an orthogonal system of functions. This is proved using the trigonometric identities $\cos nx \cos mx = \frac{1}{2}(\cos(n+m)+\cos(n-m)), \sin nx \sin mx = \frac{1}{2}(\cos(n-m)-\cos(n+m)), \sin nx \cos mx = \frac{1}{2}(\sin(n+m)+\sin(n-m))$, together with the fact that $n\pm m\neq 0$ is an integer, and so $\int_{-\pi}^\pi \cos (n\pm m) dx =0$ and $\int_{-\pi}^\pi \sin (n\pm m) dx =0$. If $n=m$, then $\cos nx \cos nx= \frac{1}{2}(\cos(2nx)+1)$ and $\sin nx \sin nx= \frac{1}{2}(1-\cos 2nx)$, so we can compute 
$
\int_{-\pi}^\pi \cos nx \cos nx dx = 
\frac{1}{2}\int_{-\pi}^\pi (\cos(2nx)+1) dx = 
\frac{1}{2} (\sin(2nx)/2n+x)\big|_{-\pi}^\pi dx = 
\pi
$ 
and
$
\int_{-\pi}^\pi \sin nx \sin nx dx = 
\frac{1}{2}\int_{-\pi}^\pi (1-\cos(2nx)) dx = 
\frac{1}{2} (x-\sin(2nx)/2n)\big|_{-\pi}^\pi dx = 
\pi
$.  These facts are used derive Euler's formulas for the Fourier coefficients.  If we multiply both sides of 
$f(x) = a_0 + \sum_{n=1}^\infty (a_n\cos nx +b_n\sin nx)$
by $\cos mx$, and then integrate term by terms, we have 
$\int_{-\pi}^{\pi} f(x)\cos(mx)dx = \int_{-\pi}^{\pi}a_0 \cos(mx)dx + \sum_{n=1}^\infty (a_n\int_{-\pi}^{\pi}\cos nx \cos(mx)dx +b_n\int_{-\pi}^{\pi}\sin nx \cos(mx)dx) = 0+a_m\pi$.  Hence $a_m = \frac{1}{\pi}\int_{-\pi}^{\pi} f(x)\cos(mx)dx$.  The other coefficients are derived similarly.

\subsection{Half-Wave Rectifier}
A half wave rectifier clips off the negative portion of a trigonometric function. The function $f(x) = \begin{cases}0&-\frac{\pi}{\omega}<x<0\\ \sin\omega x &0< x<\frac{\pi}{\omega}\end{cases}$, where $p = 2L = \frac{2\pi}{\omega}$ has had the negative portion of the sine wave clipped off (hence has passed through a half wave rectifier).  The Fourier coefficients are (using the same trig identities as above) 
$$a_0=\frac{\omega}{2\pi}\int_0^{\pi/\omega}\sin\omega x dx = -\frac{1}{2\pi}\cos\omega x \big|_0^{\pi/\omega} = \frac{1}{\pi},$$ and 
\begin{align*}
a_n 
&= \frac{\omega}{\pi}\int_{0}^{\pi/\omega}\sin \omega t \cos n\omega tdt \\
&= \frac{\omega}{2\pi}\int_{0}^{\pi/\omega}\sin(1+n)\omega t+ \sin(1-n)\omega t dt \\
&= -\frac{1}{2\pi}\left(\frac{\cos(1+n)\omega t}{(1+n)}+ \frac{\cos(1-n)\omega t}{(1-n)\omega} \right)\big|_{0}^{\pi\omega} \\
&= -\frac{1}{2\pi}\left(\frac{\cos(1+n)\pi}{(1+n)}+ \frac{\cos(1-n)\pi}{(1-n)}  - \frac{1}{1+n}-\frac{1}{1-n}\right). 
\end{align*}
If $n$ is odd, this is zero.  If $n$ is even, then $a_n = \frac{1}{2\pi}\left( \frac{2}{1+n}+\frac{2}{1-n}\right) = -\frac{2}{(n-1)(n+1)\pi}$.  You can also calculate $b_1=1/2$ and $b_n=0$ for all $n\geq 2$.  This gives the Fourier series as
$$f(x) = \frac{1}{\pi}+\frac{1}{2}\sin\omega x  - \frac{2}{\pi}\left(\frac{1}{1\cdot 3}\cos 2\omega x + \frac{1}{3\cdot 5}\cos 4\omega x +\frac{1}{5\cdot 7}\cos 6\omega x +\cdots \right).$$



\section{Even and Odd Functions}
We often use the facts that $\cos(-x)=\cos(x)$ and $\sin(-x)=-\sin(x)$ in some of the work above.  Any function $f(x)$ which satisfies $f(-x)=f(x)$ is called an even function (polynomials with only even powers of $x$ are even functions).  An odd function satisfies $f(-x)=-f(x)$ (and polynomials with only odd powers of $x$ are odd functions).  Even functions are symmetric about the $y$-axis. Odd functions are symmetric about the origin.  The Fourier coefficients of an even function are simply $a_0=\frac{1}{L}\int_0^L f(x)dx, a_n=\frac{2}{L}\int_0^L f(x)\cos \frac{n\pi x}{L}dx, b_n=0$, and the corresponding Fourier series is called a Fourier cosine series.  Similarly, for an odd function the coefficients are $a_0=0, a_n=0, b_n=\frac{2}{L}\int_0^L f(x)\sin \frac{n\pi x}{L}dx$, and the corresponding Fourier series is called a Fourier sine series. This comes because the product of two even functions is even, the product of two odd functions is even, and the product of an even and an odd function is odd. In addition, integration from $-L$ to $L$ of an odd function is zero, while integration from $-L$ to $L$ of an even function is twice the integral of $0$ to $L$.

The sawtooth wave is the function $f(x) = x+\pi$ for $-\pi<x<\pi$, and $f(x+2\pi)=f(x)$.  It can be written as the sum of an even function $f_1(x)=\pi$ and an odd function $f_2(x)=x$.  The corresponding Fourier cosine and sine series are $f_1=\pi$ and $f_2=2\left(\sin x -\frac{1}{2}\sin 2x +\frac{1}{3}\sin 3x -\frac{1}{4}\sin 4x+\cdots\right)$. Addition of series gives $f(x) = \pi+ 2\left(\sin x -\frac{1}{2}\sin 2x +\frac{1}{3}\sin 3x -\frac{1}{4}\sin 4x+\cdots\right)$. (The coefficients $b_n$ are obtained using integration by parts and $b_n = -\frac{2}{n}\cos n\pi$.)

If a function is defined on the interval $[0,L]$, then it is possible to expand the function periodically onto the interval $[-L,0]$ by either using an even expansion (reflection about the $y$ axis), or an odd expansion (reflection about the origin). Both expansions are called half-range expansions.  The Fourier series of an even half-range expansion is the Fourier cosine series, and the Fourier series of an odd half-range expansion is the Fourier sine series.  

Consider the triangle $f(x) = \begin{cases}x&0\leq x\leq L/2\\L-x&L/2\leq x\leq L\end{cases}$. The Fourier cosine series has coefficients 
$$a_0=\frac{1}{L}\int_0^{L/2} x dx+\frac{1}{L}\int_{L/2}^L (L-x) dx = \frac{1}{2},\quad\quad a_n=\frac{2}{L}\int_0^{L/2} x\cos \frac{n\pi x}{L}dx + \frac{2}{L}\int_{L/2}^L (L-x)\cos \frac{n\pi x}{L}dx.$$
Integration by parts gives 
\begin{align*}
\int_0^{L/2} x\cos \frac{n\pi x}{L}dx 
&= \left(x\frac{L}{n\pi}\sin\frac{n\pi x}{L} + \frac{L^2}{n^2\pi^2}\cos\frac{n\pi x}{L}\right)\bigg|_{0}^{L/2} \\
&= \left(\frac{L^2}{2n\pi}\sin\frac{n\pi }{2} + \frac{L^2}{n^2\pi^2}\cos\frac{n\pi }{2}\right) - \left(\frac{L^2}{n^2\pi^2}\right)
\end{align*}
and
\begin{align*}
\int_{L/2}^L (L-x)\cos \frac{n\pi x}{L}dx 
&= \left((L-x)\frac{L}{n\pi}\sin\frac{n\pi x}{L} - \frac{L^2}{n^2\pi^2}\cos\frac{n\pi x}{L}\right)\bigg|_{L/2}^L \\
&= \left( 0-\frac{L^2}{n^2\pi^2}\cos n\pi \right) 
- \left(\frac{L^2}{2n\pi}\sin\frac{n\pi}{2} - \frac{L^2}{n^2\pi^2}\cos\frac{n\pi }{2}\right).
\end{align*}
This means 
\begin{align*}
a_n 
&= \frac{2}{L}\left(\frac{L^2}{2n\pi}\sin\frac{n\pi }{2} + \frac{L^2}{n^2\pi^2}\cos\frac{n\pi }{2} - \frac{L^2}{n^2\pi^2} -\frac{L^2}{n^2\pi^2}\cos n\pi  
- \frac{L^2}{2n\pi}\sin\frac{n\pi }{2} + \frac{L^2}{n^2\pi^2}\cos\frac{n\pi }{2}\right) \\
&= \frac{2L}{n^2\pi^2}\left(2\cos\frac{n\pi }{2} -1 - \cos n\pi \right).
\end{align*}
We have $a_0=\frac{1}{2}, a_2=-8L/(2^2\pi^2), a_6= -8L/(6^2\pi^2), a_{10}=-8L/(10^2\pi^2) ,\ldots$, and $a_n=0$ for all $n$ which are odd or multiples of 4.   Hence the even expansion of $f$ has Fourier series 
$$f(x) = \frac{1}{2} -\frac{8L}{\pi^2}\left( \frac{1}{2^2}\cos \frac{2\pi x}{L} +  \frac{1}{6^2}\cos \frac{2\pi x}{L}+  \frac{1}{10^2}\cos \frac{2\pi x}{L}+\cdots\right).$$
Similar computations show that if we use a half-range odd expansion, then $b_n = \frac{4L}{n^2\pi^2}\sin\frac{n\pi}{2}$, which means $b_n = 0$ for all even $n$, and we have 
$$f(x) = \frac{4L}{\pi^2}\left( \frac{1}{1^2}\sin \frac{\pi x}{L} -  \frac{1}{3^2}\sin \frac{3\pi x}{L}+  \frac{1}{5^2}\sin \frac{5\pi x}{L}-\cdots\right).$$

\section{Identities}
Fourier series can be used to prove various identities.  For example, the Fourier series of $\sin^2 x$ is $\frac{1}{2}-\frac{1}{2}\cos(2x)$, a familiar identity.  Fourier series also give $\sin^4 x =  \frac{3}{8}-\frac{1}{2} \cos 2 x+\frac{1}{8} \cos 4 x$.  Essentially you can use Fourier series to derive a power reduction formula for any power of $\sin x$ or $\cos x$.    

In addition, Fourier series when evaluated at a point can yield interesting results. The function $f(x)=x^2$ on the interval $-1<x<1$ has Fourier coefficients $a_0=\frac{1}{3}, a_n=\frac{4}{n^2\pi^2}\cos n\pi, b_n=0$.  This means $$x^2 = \frac{1}{3}-\frac{4}{\pi^2}\left(\frac{1}{1^2}\cos \pi x -\frac{1}{2^2}\cos 2x +\frac{1}{3^2}\cos 3x-\frac{1}{4^2}\cos 4x +\cdots\right).$$  Evaluation at $0$ gives a formula for $\pi^2/12$.  Evaluation at $1/2$ and $1$ gives additional expressions involving $\pi^2$. These identities can lead to powerful ways of giving numerical approximations to $\pi$, and other numbers.

\section{Where do people use Fourier Series}
Besides mathematicians who like studying infinite series for fun, Fourier series have an extremely useful application in the telecommunications and graphics industry (cell phones, internet, land lines, JPG, MP3, radio communication, etc.).  Radio waves can be thought of as periodic vibrations of space, sent by a radio transmitter.  These vibrations are sent out in all directions, and are captured by antennae.  Your radio receiver computes Fourier integrals to compute the coefficients of the signal received.  The FCC dictates at what frequency people are allowed to broadcast.  We will discuss this more in class with an animation.

In addition, Fourier series play a major role in modeling heat transfer. Engineers use Fourier series to model the transfer of heat in jet engines, car engines, space craft, and any other device which could fail because it overheats.  You can learn more about this topic in a course on partial differential equations.  We'll take up a brief study of partial differential equations in the next chapter, and briefly show where Fourier series appear. 


}